\documentclass[12pt]{article}
\usepackage{pmmeta}
\pmcanonicalname{Java}
\pmcreated{2013-03-22 16:52:10}
\pmmodified{2013-03-22 16:52:10}
\pmowner{PrimeFan}{13766}
\pmmodifier{PrimeFan}{13766}
\pmtitle{Java}
\pmrecord{9}{39119}
\pmprivacy{1}
\pmauthor{PrimeFan}{13766}
\pmtype{Definition}
\pmcomment{trigger rebuild}
\pmclassification{msc}{68N15}
\pmdefines{Java}

\endmetadata

% this is the default PlanetMath preamble.  as your knowledge
% of TeX increases, you will probably want to edit this, but
% it should be fine as is for beginners.

% almost certainly you want these
\usepackage{amssymb}
\usepackage{amsmath}
\usepackage{amsfonts}

% used for TeXing text within eps files
%\usepackage{psfrag}
% need this for including graphics (\includegraphics)
%\usepackage{graphicx}
% for neatly defining theorems and propositions
%\usepackage{amsthm}
% making logically defined graphics
%%%\usepackage{xypic}

% there are many more packages, add them here as you need them

% define commands here

\begin{document}
{\em Java} is a software platform intended to modernize programming on a universal scale.  The chief inventor was James Gosling working for Sun Microsystems which has retained the governance of the Java platform specifications.  A strong community of other companies and users has developed and affected the development chiefly IBM and Apple.  

Gosling's hope for development on the platform was encompassed in the slogan
\begin{quote}
``Write once, run anywhere.''
\end{quote}
Unfortunately, the realization has often followed the mocking slogan:
\begin{quote}
``Write once, debug everywhere.''
\end{quote}
The name Java is a reference to coffee but was not the original name 
desired by Gosling.

\section{The platform}

The platform is comprised of several levels of abstraction.
\begin{itemize}
\item Every Java platform operates with a core kernel which is designed and implemented for specific physical settings.  There are Java kernels
targeted for multiprocessor Sparc and Itanium server environments all the way down to StrongArm and even Smart-Card chips.  IBM has even prototype processors equipped to operate as  solid state Java kernels.  Access to this layer from the other platforms is achieved through \texttt{jave.lang} packages as well as through the Java Native Interface.

The simplest method to create a kernel is to build an interpreter in the
language and tools of a given operating system, such as Microsoft Windows, 
Apple's OS-X, and various Linux and Unix distributions.  This unnecessary 
layer is generally responsible for slower performance on time critical tasks 
and may be responsible for much of the limited adoption of platform for large scale home or office applications.

The most notable divergence of the kernel from C style applications is 
garbage collection.  Memory in Java is managed completely by the kernel and
so programmers cannot use dangerous tricks such as incrementing or decrementing pointers beyond the bounds of the array or structure allocated to pointer.
However, by retaining deletion controls, some critical optimizations may not
be able release resources when desired.


\item The second layer is the programming language and its suite of standard
packages.  The language uses the C syntax but unlike C++ makes an explicit
use of object oriented structures.  The concept of exceptions, experimentally introduced in C++, is given a robust form thus removing the need to 
check returns for errors.  An error in Java reverses the execution direction
of the thread ``throwing'' exception to the calling functions until some 
layer is capable of handling the error.  An unexpected result of this is that
some programs now use the exception mechanism as a valid exit strategy from
a loop of recursion thus in effect providing for an unbounded number of types
as the result of a method invocation.    

The language has single inheritance but allows for polymorphism through interface structures.  This avoids the danger of C++ multiple inheritance.  

The various packages, for example: \texttt{java.util}, \texttt{java.awt}, \texttt{java.net}, etc. provide uniform access to standard features of most
hardware systems.  For example, data structures such as lists, trees, queue, etc. are provided in the utility package.  AWT controls standard windowing controls which allows for elegant and simply built graphical user interfaces.  Packages such as the net package provide very direct methods to access internet controls (TCP/IP stacks) and provide conversions between 
various formats on the web.  This makes network applications very favorable
as there is no concern over which operating systems are linked together
through the programs.

\item At the most abstract layer to the platform are the extended packages.
These include packages to control databases, XML utilities, 3-D packages, 
speech and natural language, and perhaps the most comprehensive dedication to
accessibility and internationalization.
\end{itemize}

\section{The standards}

Having developed the Java platform from the ground up there was a rare opportunity to establish conventions to form a uniform development look
which would promote ease of development.  Data types are named with a capital
letter for each new word or initial.  For example, the data structure
encoding a list is written \texttt{List}.  The exception to this rule
are the primitive data types of integers, floating points, etc.  These
are written \texttt{int}, \texttt{long}, \texttt{boolean}, etc.  

In contrast, variables (references) begin with a lower case but may have
upper case letters for subsequent words or initials.  For example,
an instance of an object \texttt{LargeInteger} could be named 
\texttt{largeInt}.  

Unlike the Microsoft Hungarian notation, references are not labeled with
short hand letters revealing their type.  Because Java is a strongly typed
programming language, much of the type problems of C/C++ solved by Hungarian
notation is unnecessary as the problems are caught at compile time.  The trade off is the inflexibility of programming in that data must be explicitly cast 
 to the appropriate data type even when provably clear in the 
program.  Later languages, such as Python, are in sharp contrast to Java in this
regard.


At larger level, the names of new packages carry with them the website names of the developers.  For example \texttt{com.sun.java.swing} or \texttt{com.ibm.java.speech} are packages from Sun Microsystems and IBM, as is easily detected in the package names.  Linking between libraries is thus simply a matter of identifying the package name and is not affected by separate MAKE files etc. as is common in C/C++ settings.

\section{Mathematical and scientific applications}

In the world of mathematical and scientific computation, the use of Java is limited.  Most theoretical applications are encumbered by unnecessary user interfaces and the strong need to optimize makes garbage collects and especially interpreters very unfavorable.  The one place were the Java platform is poised for use on large scale mathematical projects is in distributed computing problems such as searches for primes etc.   

\subsection{External links}

\PMlinkexternal{Mirek's Java Cellebration v.1.50}{http://www.mirwoj.opus.chelm.pl/ca/mjcell/mjcell.html}
%%%%%
%%%%%
\end{document}
