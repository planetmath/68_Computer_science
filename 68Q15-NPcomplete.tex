\documentclass[12pt]{article}
\usepackage{pmmeta}
\pmcanonicalname{NPcomplete}
\pmcreated{2013-03-22 13:01:49}
\pmmodified{2013-03-22 13:01:49}
\pmowner{Henry}{455}
\pmmodifier{Henry}{455}
\pmtitle{NP-complete}
\pmrecord{4}{33429}
\pmprivacy{1}
\pmauthor{Henry}{455}
\pmtype{Definition}
\pmcomment{trigger rebuild}
\pmclassification{msc}{68Q15}
\pmsynonym{NP complete}{NPcomplete}
\pmsynonym{NP hard}{NPcomplete}
\pmdefines{NP-hard}

\endmetadata

% this is the default PlanetMath preamble.  as your knowledge
% of TeX increases, you will probably want to edit this, but
% it should be fine as is for beginners.

% almost certainly you want these
\usepackage{amssymb}
\usepackage{amsmath}
\usepackage{amsfonts}

% used for TeXing text within eps files
%\usepackage{psfrag}
% need this for including graphics (\includegraphics)
%\usepackage{graphicx}
% for neatly defining theorems and propositions
%\usepackage{amsthm}
% making logically defined graphics
%%%\usepackage{xypic}

% there are many more packages, add them here as you need them

% define commands here
%\PMlinkescapeword{theory}
\begin{document}
A problem $\pi\in\mathcal{NP}$ is $\mathcal{NP}$ \emph{\PMlinkescapetext{complete}} if for any $\pi^\prime\in\mathcal{NP}$ there is a Cook reduction of $\pi^\prime$ to $\pi$.  Hence if $\pi\in\mathcal{P}$ then every $\mathcal{NP}$ problem would be in $\mathcal{P}$.  A slightly stronger definition requires a Karp reduction or Karp reduction of corresponding decision problems as appropriate.

A search problem $R$ is \emph{$\mathcal{NP}$ hard} if for any $R^\prime\in\mathcal{NP}$ there is a Levin reduction of $R^\prime$ to $R$.
%%%%%
%%%%%
\end{document}
