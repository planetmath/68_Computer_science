\documentclass[12pt]{article}
\usepackage{pmmeta}
\pmcanonicalname{PolynomialHierarchy}
\pmcreated{2013-03-22 13:02:11}
\pmmodified{2013-03-22 13:02:11}
\pmowner{Henry}{455}
\pmmodifier{Henry}{455}
\pmtitle{polynomial hierarchy}
\pmrecord{6}{33437}
\pmprivacy{1}
\pmauthor{Henry}{455}
\pmtype{Definition}
\pmcomment{trigger rebuild}
\pmclassification{msc}{68Q15}
\pmdefines{PH}

\endmetadata

% this is the default PlanetMath preamble.  as your knowledge
% of TeX increases, you will probably want to edit this, but
% it should be fine as is for beginners.

% almost certainly you want these
\usepackage{amssymb}
\usepackage{amsmath}
\usepackage{amsfonts}

% used for TeXing text within eps files
%\usepackage{psfrag}
% need this for including graphics (\includegraphics)
%\usepackage{graphicx}
% for neatly defining theorems and propositions
%\usepackage{amsthm}
% making logically defined graphics
%%%\usepackage{xypic}

% there are many more packages, add them here as you need them

% define commands here
%\PMlinkescapeword{theory}
\begin{document}
The \emph{polynomial hierarchy} is a hierarchy of complexity classes generalizing the relationship between $\mathcal{P}$ and $\mathcal{NP}$.

We let $\Sigma^p_0=\Pi^p_0=\Delta^p_0=\mathcal{P}$, then $\Delta^p_{i+1}=\mathcal{P}^{\Sigma^p_i}$ and $\Sigma^p_{i+1}=\mathcal{NP}^{\Sigma^p_i}$.  Define $\Pi^p_i$ to be  $co\Sigma^p_i$.

For instance $\Sigma^p_1=\mathcal{NP}^\mathcal{P}=\mathcal{NP}$.

The complexity class $\mathcal{PH}=\bigcup_{i\in\mathbb{N}} \Sigma^p_i$.

The polynomial hierarchy is closely related to the arithmetical hierarchy; indeed, an alternate definition is almost identical to the definition of the arithmetical hierarchy but stricter rules on what quantifiers can be used.

When there is no risk of confusion with the arithmetical hierarchy, the superscript $p$ can be dropped.
%%%%%
%%%%%
\end{document}
