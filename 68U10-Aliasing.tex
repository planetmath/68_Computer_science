\documentclass[12pt]{article}
\usepackage{pmmeta}
\pmcanonicalname{Aliasing}
\pmcreated{2013-03-22 12:04:23}
\pmmodified{2013-03-22 12:04:23}
\pmowner{akrowne}{2}
\pmmodifier{akrowne}{2}
\pmtitle{aliasing}
\pmrecord{8}{31142}
\pmprivacy{1}
\pmauthor{akrowne}{2}
\pmtype{Definition}
\pmcomment{trigger rebuild}
\pmclassification{msc}{68U10}
\pmclassification{msc}{94A08}

\endmetadata

\usepackage{amssymb}
\usepackage{amsmath}
\usepackage{amsfonts}
\usepackage{graphicx}
%%%\usepackage{xypic}
\begin{document}
{\bf Aliasing}

Used in the context of processing digitized signals (e.g. audio) and images (e.g. video), aliasing describes the effect of undersampling during digitization which can generate a false (apparent) low frequency for signals, or staircase steps along edges in images (jaggies.) Aliasing can be avoided by an antialiasing (analogue) low-pass filter, before sampling. The term antialiasing is also in use for \emph{a posteriori} signal smoothing intended to remove the effect.

{\bf References}

\begin{itemize}
\item Based on content from The Data Analysis Briefbook (\PMlinkexternal{http://rkb.home.cern.ch/rkb/titleA.html}{http://rkb.home.cern.ch/rkb/titleA.html})
\end{itemize}
%%%%%
%%%%%
%%%%%
\end{document}
