\documentclass[12pt]{article}
\usepackage{pmmeta}
\pmcanonicalname{KolmogorovComplexity}
\pmcreated{2013-03-22 13:43:41}
\pmmodified{2013-03-22 13:43:41}
\pmowner{tromp}{1913}
\pmmodifier{tromp}{1913}
\pmtitle{Kolmogorov complexity}
\pmrecord{9}{34414}
\pmprivacy{1}
\pmauthor{tromp}{1913}
\pmtype{Topic}
\pmcomment{trigger rebuild}
\pmclassification{msc}{68Q30}
\pmsynonym{algorithmic information theory}{KolmogorovComplexity}
\pmsynonym{algorithmic entropy}{KolmogorovComplexity}
%\pmkeywords{binary string}
%\pmkeywords{information}
%\pmkeywords{randomness}
%\pmkeywords{entropy}
%\pmkeywords{universal}
%\pmkeywords{invariance}

\endmetadata

% this is the default PlanetMath preamble.  as your knowledge
% of TeX increases, you will probably want to edit this, but
% it should be fine as is for beginners.

% almost certainly you want these
\usepackage{amssymb}
\usepackage{amsmath}
\usepackage{amsfonts}

% used for TeXing text within eps files
%\usepackage{psfrag}
% need this for including graphics (\includegraphics)
%\usepackage{graphicx}
% for neatly defining theorems and propositions
%\usepackage{amsthm}
% making logically defined graphics
%%%\usepackage{xypic}

% there are many more packages, add them here as you need them

% define commands here
\begin{document}
\PMlinkescapeword{binary}
\PMlinkescapeword{measure}
\PMlinkescapeword{information}

Consider flipping a coin 50 times to obtain the binary string
000101000001010100010000010101000100000001010000010.
Can we call this random? The string has rather an abundance of 0s,
and on closer inspection every other bit is 0. We wouldn't expect even
a biased coin to come up with such a pattern. Still, this string
has probability $2^{-50}$, just like any other binary string of the same length,
so how can we call it any less random?

{\em Kolmogorov Complexity} provides an answer to these questions in the form
of a measure of information content in individual objects. Objects with low
information content may be considered non-random.
The topic was founded in the 1960s independently by three people:
Ray Solomonoff, Andrei Kolmogorov, and Gregory Chaitin.

See Kolmogorov complexity function and invariance theorem for more details.
%%%%%
%%%%%
\end{document}
