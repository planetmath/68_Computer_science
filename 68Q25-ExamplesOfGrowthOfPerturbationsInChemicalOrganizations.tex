\documentclass[12pt]{article}
\usepackage{pmmeta}
\pmcanonicalname{ExamplesOfGrowthOfPerturbationsInChemicalOrganizations}
\pmcreated{2014-05-31 15:55:20}
\pmmodified{2014-05-31 15:55:20}
\pmowner{rspuzio}{6075}
\pmmodifier{rspuzio}{6075}
\pmtitle{examples of growth of perturbations in chemical organizations}
\pmrecord{33}{88110}
\pmprivacy{1}
\pmauthor{rspuzio}{6075}
\pmtype{Example}

\endmetadata

% this is the default PlanetMath preamble.  as your knowledge
% of TeX increases, you will probably want to edit this, but
% it should be fine as is for beginners.

% almost certainly you want these
\usepackage{amssymb}
\usepackage{amsmath}
\usepackage{amsfonts}

% need this for including graphics (\includegraphics)
\usepackage{graphicx}
% for neatly defining theorems and propositions
\usepackage{amsthm}

% making logically defined graphics
%\usepackage{xypic}
% used for TeXing text within eps files
%\usepackage{psfrag}

% there are many more packages, add them here as you need them

% define commands here

\newtheorem{thm}{Theorem}
\begin{document}
We will examine several simple examples of chemical systems where we start with one species 
of molecules (or a closed subset of species) then intoruduce a small perturbation and evolve 
the system using mass action dynamics.  We want to know whether this perturbation will grow
and, if so, at what rate.  Ultimately, we would like to link the behavor to some feature of
the reaction system, perhaps related to Rosen's theory of M-R systems.

To get started, consider a trivial case, $A + B \to B + B$.  The system of equations which
describes this system is:
\begin{align*}
 \frac{dx}{dt} &= -kxy \\
 \frac{dy}{dt} &= kxy
\end{align*}
It is easy enough to solve this system.  We begin by noting that
$\frac{d}{dt} (x + y) =0$, hence $x + y = x_0 + y_0$.
Substituting this back in to the second equation, we conclude
that
\[
 \frac{dy}{dt} = k(x_0 + y_0 - y)(y) .
\]
This equation can readily be solved to yield the implicit solution
\[
 kt = \frac{1}{x_0 + y_0} \log \frac{y}{x_0 + y_0 - y} \cdot \frac{x_0}{y_0}
\]
which can be solved to produce the explicit solution
\[
 y = x_0 + y_0 - \frac{x_0 + y_0}{1 + \frac {y_0}{x_0} \exp(k (x_0 + y_0) t)}.
\]
Looking at the solution, we see that it starts out at $y = y_0$ and grows
towards $y = x_0 + y_0$ as $t \to \infty$.  This is as we expect --- as
time goes on, whatever A's there are left react with B's to turn into B's
until we are left with nothing but B's.

If we suppose that, at the initial time $t = 0$, there is only a tiny 
proportion of B's, i.e. $y_0 \ll x_0$, then we may make an expansion
of the fraction and conclude that $y$ grows exponentially for small
values of $t$:
\[
 \frac{1}{1 + \frac {y_0}{x_0} \exp(k (x_0 + y_0) t)} \approx
 1 - \frac {y_0}{x_0} \exp(k (x_0 + y_0) t)
\]
\[
 y \approx \frac {y_0}{x_0} \exp(k (x_0 + y_0) t)
\]
We can also come to this conclusion by bounding $y$ without solving the
equation first.  For a simple equation like this which is readily solved,
this is hardly needed but, for larger, more complicated equations, it
becomes important and this simple example can serve as a illustration of
the general technique.

\begin{thm}
Let $C$ be a real number such that $0 < C < 1$ and let $x_0$ and $y_0$ be 
strictly positive real numbers such that $0 < y_0 < C (x_0 + y_0)$.
Set $t_1 = \frac{1}{k (x_0 + y_0)} \log C \frac{x_0 + y_0}{y_0}$.  Then 
there exists a function $f \colon [0,t_1) \to [0, C (x_0 + y_0))$ such that
$f$ satisfies the differential equation
\[
 \frac{df(t)}{dt} = k (x_0 + y_0 - f(t)) f(t).
\]
and, for all $t \in [0,t_1)$,
\[
 y_0 \exp ((1 - C) k (x_0 + y_0) t) \le f(t) \le y_0 \exp (k (x_0 + y_0) t)  .
\]
\end{thm}

\begin{proof}
By the existence theorem, there exists a positive real number 
$t_{0}$ and a function  $f \colon [0,t_{0}) \to \mathbb{R}$
such that $f(0) = y_0$ and $f$ satifies the differential equation. 
Since $f(0) = y_0 < C (x_0 + y_0)$, by continuity there exists 
a positive real number $t_2$ and a function $f \colon [0,t_{0}) \to [0, C (x_0 + y_0))$
which satisfies the same differential equation with the same 
initial condition.  Furthermore, we assume that $t_2$ is maximal.

Starting with this condition $f(t) < C (x_0 + y_0)$ and
doing some algebra, we conclude that
\[
 (1 - C) k (x_0 + y_0) \le \frac{1}{y} \frac{dy}{dt} \le k (x_0 + y_0)
\]
Now, $\frac{1}{y} \frac{dy}{dt} = \frac{d}{dt} (\log y)$ so, by the mean
value theorem, we conclude that
\[
 (1 - C) k (x_0 + y_0) t \le \log \frac{y}{y_0} \le k (x_0 + y_0) t .
\]
Exponentiating,
\[
 y_0 \exp ((1 - C) k (x_0 + y_0) t) \le y \le y_0 \exp (k (x_0 + y_0) t)  .
\]


We can ensure that the bound on $y$ is satisfied if the condition
$C (x_0 + y_0) \ge y_0 \exp (k (x_0 + y_0) t)$ is met, which amounts to
demanding that $0 \le t \le t_1$ where $t_1 = \frac{1}{k (x_0 + y_0)}
\log C \frac{x_0 + y_0}{y_0}$.

\end{proof}
\end{document}
