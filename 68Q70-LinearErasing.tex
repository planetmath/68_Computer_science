\documentclass[12pt]{article}
\usepackage{pmmeta}
\pmcanonicalname{LinearErasing}
\pmcreated{2013-03-22 18:58:54}
\pmmodified{2013-03-22 18:58:54}
\pmowner{CWoo}{3771}
\pmmodifier{CWoo}{3771}
\pmtitle{linear erasing}
\pmrecord{8}{41846}
\pmprivacy{1}
\pmauthor{CWoo}{3771}
\pmtype{Definition}
\pmcomment{trigger rebuild}
\pmclassification{msc}{68Q70}
\pmsynonym{limited erasing}{LinearErasing}
\pmrelated{RestrictedHomomorphism}

\usepackage{amssymb,amscd}
\usepackage{amsmath}
\usepackage{amsfonts}
\usepackage{mathrsfs}

% used for TeXing text within eps files
%\usepackage{psfrag}
% need this for including graphics (\includegraphics)
%\usepackage{graphicx}
% for neatly defining theorems and propositions
\usepackage{amsthm}
% making logically defined graphics
%%\usepackage{xypic}
\usepackage{pst-plot}

% define commands here
\newcommand*{\abs}[1]{\left\lvert #1\right\rvert}
\newtheorem{prop}{Proposition}
\newtheorem{thm}{Theorem}
\newtheorem{ex}{Example}
\newcommand{\real}{\mathbb{R}}
\newcommand{\pdiff}[2]{\frac{\partial #1}{\partial #2}}
\newcommand{\mpdiff}[3]{\frac{\partial^#1 #2}{\partial #3^#1}}
\begin{document}
It is well-known that, among all of the language families in the Chomsky hierarchy, the family $\mathscr{S}$ of context-sensitive languages is the only one that is not closed under arbitrary homomorphisms.  Nevertheless, $\mathscr{S}$ is shown to be closed under a more restricted class of homomorphisms, namely the $\lambda$-free homomorphisms.  Question: can we enlarge this class of homomorphisms so that $\mathscr{S}$ is still closed under the larger class?  The answer is yes.

\textbf{Definition}.  Let $L$ be a language over an alphabet $\Sigma$, $h$ a homomorphism over $\Sigma$, and $k$ a non-negative integer.  $h$ is said to be a \emph{$k$-linear erasing} on $L$ if for any word $u\in L$, we have $$|u|\le k |h(u)|,$$
where $|u|$ stands for the length of $u$.

It is clear that if $h$ is a $k$-linear erasing on $L$, then it is a $m$-linear erasing for any $m\ge k$.  Also, if $h$ is a $0$-linear erasing on $L$, then $L$ is either $\lbrace \lambda\rbrace$, or the empty set $\varnothing$.  In addition, if $h$ is a $k$-linear erasing on $L$, and $L'\subseteq L$, then it is a $k$-linear erasing on $L'$.  Consequently, any $\lambda$-free homomorphism is a $k$-linear erasing on any $L$ over $\Sigma$, for any $k\ge 1$.

However, the notion of linear erasing is language dependent.  For example, let $\Sigma = \lbrace a,b,c\rbrace$.  Let $L_1 = \lbrace a^nb^n \mid n\ge 0\rbrace$ and $L_2= \lbrace a^nc^n \mid n\ge 0\rbrace$.  Suppose $h$ is the homomorphism on $\Sigma^*$ with $h(a)=\lambda$, $h(b)=b^2$ and $h(c)=c$.  Then $h$ is a $1$-linear erasing on $L_1$, and a $2$-linear erasing on $L_2$.

\textbf{Definition}  Let $\mathscr{L}$ be a family of languages over $\Sigma$.  Then $\mathscr{L}$ is said to be \emph{closed under linear erasing} if for any $L\in \mathscr{L}$, and any homomorphism $h$ which is a $k$-linear erasing on $L$ for some $k\ge 0$, then $h(L)\in \mathscr{L}$.

Clearly, if $\mathscr{L}$ is closed under homomorphism, it is closed under linear erasing, and thus the families of \PMlinkname{regular}{RegularLanguage}, context-free, and type-0 languages are all closed under linear erasing.  We also have the following:
\begin{thm} The family $\mathscr{S}$ of context-sensitive languages is closed under linear erasing. \end{thm}

\textbf{Remark}.  The theorem above can be generalized.  Call a substitution $s$ over $\Sigma$ a $k$-linear erasing on a language $L$ if $|u|\le k|v|$ for any $v\in s(u)$.  If $L$ is context-sensitive such that $s(u)$ is context-sensitive for each $u\in L$, then $s(L)$ is context-sensitive provided that $s$ is a $k$-linear erasing on $L$.

\begin{thebibliography}{9}
\bibitem{AS} A. Salomaa, {\em Formal Languages}, Academic Press, New York (1973).
\bibitem{hu} J.E. Hopcroft, J.D. Ullman, {\em Formal Languages and Their Relation to Automata}, Addison-Wesley, (1969).
\end{thebibliography}
%%%%%
%%%%%
\end{document}
