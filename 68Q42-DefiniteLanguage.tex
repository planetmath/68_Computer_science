\documentclass[12pt]{article}
\usepackage{pmmeta}
\pmcanonicalname{DefiniteLanguage}
\pmcreated{2013-03-22 18:58:57}
\pmmodified{2013-03-22 18:58:57}
\pmowner{CWoo}{3771}
\pmmodifier{CWoo}{3771}
\pmtitle{definite language}
\pmrecord{10}{41847}
\pmprivacy{1}
\pmauthor{CWoo}{3771}
\pmtype{Definition}
\pmcomment{trigger rebuild}
\pmclassification{msc}{68Q42}
\pmclassification{msc}{68Q45}

\usepackage{amssymb,amscd}
\usepackage{amsmath}
\usepackage{amsfonts}
\usepackage{mathrsfs}

% used for TeXing text within eps files
%\usepackage{psfrag}
% need this for including graphics (\includegraphics)
%\usepackage{graphicx}
% for neatly defining theorems and propositions
\usepackage{amsthm}
% making logically defined graphics
%%\usepackage{xypic}
\usepackage{pst-plot}

% define commands here
\newcommand*{\abs}[1]{\left\lvert #1\right\rvert}
\newtheorem{prop}{Proposition}
\newtheorem{thm}{Theorem}
\newtheorem{ex}{Example}
\newcommand{\real}{\mathbb{R}}
\newcommand{\pdiff}[2]{\frac{\partial #1}{\partial #2}}
\newcommand{\mpdiff}[3]{\frac{\partial^#1 #2}{\partial #3^#1}}
\newtheorem{cor}{Corollary}
\begin{document}
A language is \emph{definite} if the determination of whether a word belongs to the language completely depends on its suffix of a fixed length.  Formally,

\textbf{Definition}.  Let $L$ be a language over an alphabet $\Sigma$.  Then $L$ is \emph{definite} if there is a non-negative integer $k$ such that for any word $u$ over $\Sigma$ with $|u|\ge k$, its suffix $v$ of length $k$ is in $L$ iff $u$ is in $L$.

Note that if $k=0$, then $L$ is either $\Sigma^*$ or $\varnothing$.

A definite language has the following characterization:
\begin{prop}  $L$ is definite iff there are finite languages $L_1$ and $L_2$ such that $$L=L_1\cup \Sigma^* L_2.$$
\end{prop}
\begin{proof}
Suppose first that $L$ is definite.  Let $L_1$ be the subset of $L$ containing all words with length less than $k$ and $L_2$ the subset of $L$ containing all words of length $k$.  The case when $k=0$ is already mentioned in the note above.  If $k=1$, $L_1$ is either $\lbrace \lambda \rbrace$ or $\varnothing$, depending on whether or not $\lambda\in L$.  In any case, $L_1$ and $L_2$ are both finite.  We verify that $L=L_1\cup \Sigma^*L_2$.  If $u\in L$ and $u$ has length less than $k$, then $u\in L_1$.  Otherwise, it has length at least $k$.  By the definiteness of $L$, its suffix $v$ of length $k$ is in $L$, and thus is in $L_2$.  Therefore $u\in \Sigma^*L_2$ as a result.  This shows that $L\subseteq L_1\cup \Sigma^*L_2$.  On the other hand, suppose $u \in L_1\cup \Sigma^*L_2$.  If $|u|<k$, then $u\in L_1 \subseteq L$.  If $|u|\ge k$, write $u=wv$ where $v$ is the suffix with length $k$.  Then $v\in L_2$, which means that $u\in L$ since $L$ is definite.

Now suppose $L=L_1\cup \Sigma^* L_2$ with $L_1,L_2$ finite.  Let $k$ be the maximum of lengths of words in $L_1\cup L_2$, plus $1$.  $k$ is well-defined since $L_1\cup L_2$ is finite.  Suppose $u$ is a word in $\Sigma^*$ with $|u|\ge k$.  Then $u\notin L_1$.  Let $v$ be the suffix of $u$ with $|v|=k$.  Then $v\notin L_1$ likewise.  If $v\in L$, then $v\in \Sigma^*L_2$ and hence $u\in \Sigma^*L_2$ as well.  If $u\in L$, then $u\in \Sigma^*L_2$, so that $u$ has a suffix $w\in L_2$.  Since $|w|<k$, $w$ is also a suffix of $v$, and therefore $v\in \Sigma^*L_2 \subseteq L$ as a result.  This shows that $L$ is definite.
\end{proof}

From this characterization, we see that finite languages are special cases of definite languages.  We also see that
\begin{cor} Every definite language is a regular language. \end{cor}

With regard to the closure properties of definite languages, we have the following: let $\mathscr{D}$ be the family of definite languages, then $\mathscr{D}$ is closed under union and complementation, and therefore any Boolean operations.  However, $\mathscr{D}$ is not closed under concatenation, Kleene star, and reversal.

\textbf{Remark}.  In fact, every definite language is locally testable.  The converse, however, is not true.

\begin{thebibliography}{9}
\bibitem{AS} A. Salomaa, {\em Formal Languages}, Academic Press, New York (1973).
\end{thebibliography}
%%%%%
%%%%%
\end{document}
