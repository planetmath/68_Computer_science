\documentclass[12pt]{article}
\usepackage{pmmeta}
\pmcanonicalname{FORTRAN}
\pmcreated{2013-03-22 16:46:38}
\pmmodified{2013-03-22 16:46:38}
\pmowner{PrimeFan}{13766}
\pmmodifier{PrimeFan}{13766}
\pmtitle{FORTRAN}
\pmrecord{9}{39007}
\pmprivacy{1}
\pmauthor{PrimeFan}{13766}
\pmtype{Topic}
\pmcomment{trigger rebuild}
\pmclassification{msc}{68W30}
\pmclassification{msc}{68N15}

\endmetadata

% this is the default PlanetMath preamble.  as your knowledge
% of TeX increases, you will probably want to edit this, but
% it should be fine as is for beginners.

% almost certainly you want these
\usepackage{amssymb}
\usepackage{amsmath}
\usepackage{amsfonts}

% used for TeXing text within eps files
%\usepackage{psfrag}
% need this for including graphics (\includegraphics)
%\usepackage{graphicx}
% for neatly defining theorems and propositions
%\usepackage{amsthm}
% making logically defined graphics
%%%\usepackage{xypic}

% there are many more packages, add them here as you need them

% define commands here

\begin{document}
{\em FORTRAN}{\footnote{Acronym that comes from \emph{For-mula Tran-slating.}}} (or {\em ForTran} or {\em Fortran}) is a computer programming language developed by IBM in the 1950s with a focus on scientific and engineering applications. FORTRAN is still in use today in the sciences despite the dominance of \PMlinkname{C++}{C} for most general applications and Mathematica and Maple for algebra applications. Sloane's On-Line Encyclopedia of Integer Sequences does not include FORTRAN source code but provides links to it. The current version is FORTRAN 2003; an international group of corporations and programmers is working on FORTRAN 2008. % update when applicable: a beta version is now available

The following FORTRAN program takes two integers as inputs and outputs their greatest common divisor using Euclid's algorithm. It requires positive integers as inputs. NB is just a variable name that has nothing to do with Zentrums. The function NGCD is defined after the main program. It was written by Wikipedia user Rwwww:

\begin{verbatim}
*     euclid.f (FORTRAN 77)
*     Find greatest common divisor using the Euclidean algorithm
*     Written by: Wikipedia User:Rwwww

      PROGRAM EUCLID
        PRINT *, 'A?'
        READ *, NA
        IF (NA.LE.0) THEN
          PRINT *, 'A must be a positive integer.'
          STOP
        END IF
        PRINT *, 'B?'
        READ *, NB
        IF (NB.LE.0) THEN
          PRINT *, 'B must be a positive integer.'
          STOP
        END IF
        PRINT *, 'The GCD of', NA, ' and', NB, ' is', NGCD(NA, NB), '.'
        STOP
      END

      FUNCTION NGCD(NA, NB)
        IA = NA
        IB = NB
    1   IF (IB.NE.0) THEN
          ITEMP = IA
          IA = IB
          IB = MOD(ITEMP, IB)
          GOTO 1
        END IF
        NGCD = IA
        RETURN
      END
\end{verbatim}
%%%%%
%%%%%
\end{document}
