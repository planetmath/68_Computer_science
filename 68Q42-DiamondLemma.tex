\documentclass[12pt]{article}
\usepackage{pmmeta}
\pmcanonicalname{DiamondLemma}
\pmcreated{2013-03-22 15:23:51}
\pmmodified{2013-03-22 15:23:51}
\pmowner{CWoo}{3771}
\pmmodifier{CWoo}{3771}
\pmtitle{diamond lemma}
\pmrecord{9}{37235}
\pmprivacy{1}
\pmauthor{CWoo}{3771}
\pmtype{Topic}
\pmcomment{trigger rebuild}
\pmclassification{msc}{68Q42}
\pmclassification{msc}{03C05}
\pmrelated{TerminatingReduction}
\pmrelated{NormalizingReduction}
\pmdefines{reduction}

\endmetadata

% almost certainly you want these
\usepackage{amssymb}
\usepackage{amsmath}
\usepackage{amsfonts}

% used for TeXing text within eps files
%\usepackage{psfrag}
% need this for including graphics (\includegraphics)
%\usepackage{graphicx}
% for neatly defining theorems and propositions
\usepackage{amsthm}
\newtheorem{thm}{Theorem}
\newtheorem{lem}{Lemma}
\newtheorem{alg}{Algorithm}
\theoremstyle{definition}
\newtheorem{definition}{Definition}


% making logically defined graphics
\usepackage[all]{xy}

% there are many more packages, add them here as you need them

% define commands here

\providecommand{\cprime}{\ensuremath{{}^\prime}}
\providecommand*{\Dash}{%
   \hspace*{0.166667em}\textemdash\hspace{0.166667em}%
}
\providecommand*{\Ldash}{%
   \hspace{0.166667em}\textemdash\hspace*{0.166667em}%
}
\newcommand*{\setOf}[2]{%
   \left\{ \, #1 \,\,\vrule\,\, #2 \, \right\}%
}
\begin{document}
The diamond lemmas constitute a \PMlinkescapetext{class} of results about the existence 
of a unique normal form. Diamond lemmas exist in many diverse areas of 
mathematics, and as a result their technical contents can be quite 
different, but they are easily recognisable from their overall 
\PMlinkescapetext{structure} and basic idea\Dash the ``diamond'' condition 
from which they inherit their name.

\PMlinkescapeword{Theorem}
\PMlinkescapeword{Lemma}
\PMlinkescapeword{clear}
\PMlinkescapeword{state}
\PMlinkescapeword{origin}
\PMlinkescapeword{even}


\section{Newman's Diamond Lemma}

The basic diamond lemma, that of Newman~\cite{Newman}, is today most 
easily presented in \PMlinkescapetext{terms} of binary relations.

\begin{thm} \label{S:NDL}
  Let $X$ be a set and let $\rightarrow$ be a binary relation on $X$. 
  If $\rightarrow$ is terminating, then the following conditions are equivalent:
  \begin{enumerate}
    \item \label{Cond:Diamond}
      For all \(a,b,c \in X\) such that \(a \rightarrow b\) and \(a 
      \rightarrow c\), then $b$ and $c$ are joinable.
    \item \label{Cond:UniqueNF}
      Every \(a \in X\) has a unique normal form.
  \end{enumerate}
\end{thm}

Condition~\ref{Cond:Diamond} can be graphically drawn as
$$
  \xymatrix@-2ex{
    & a \ar[dl] \ar[dr] \\
    b \ar[dr]^>>>>{*} & & c \ar[dl]_>>>>{*} \\
    & d
  }
$$
where $\stackrel{*}{\rightarrow}$ denotes the reflexive transitive closure of $\rightarrow$.  This is the ``diamond'' from which the result derives its 
name\Dash not crystallised carbon, but merely a square standing on one 
corner. In relation to this picture, what the condition says is that 
whenever one has \(a,b,c \in X\) forming the top two sides of such a 
diagram, there is a fourth corner \(d \in X\) which completes the 
diamond.

\medskip

The typical case to which one applies this theorem is that $X$ is a 
set of formal terms on which one wishes to define an equivalence 
relation $\sim$. The $\rightarrow$ relation encodes a set of 
``elementary'' equivalences\Ldash an example in standard algebra of 
such an elementary equivalence might be the instance 
\(a(b +\nobreak c) \rightarrow ab + ac\) of the distributivity 
axiom\Dash that generate the wanted $\sim$. It is 
techincally possible to define the equivalence $\sim$ as the 
reflexive--transitive--symmetric closure of $\rightarrow$, but 
since $X$ is typically infinite that definition does not lead to 
an effective method for determining whether \(x \sim y\). However 
when the relation $\rightarrow$ satisfies the conditions in 
Theorem~\ref{S:NDL}, there does exist an algorithm which given 
$\rightarrow$ determines whether two arbitrary elements are related 
by $\sim$.

In a computational setting, the interpretation of \(x \rightarrow y\) 
is usually that there exists a ``one-step simplification'' that 
converts the expression $x$ to the expression $y$. The corresponding 
interpretation of \(x \stackrel{*}{\rightarrow} y\) is thus that 
there exists a finite sequence of ``simplifications'' that converts 
$x$ to $y$. The formal term used for ``simplifications'' in this 
sense is \emph{reductions}, so \(x \stackrel{*}{\rightarrow} y\) is 
read as ``$x$ reduces to $y$'' or ``$x$ can be reduced to $y$''. 

%\begin{definition}
%  Let $X$ be a set and $\rightarrow$ a binary relation on $X$. An 
%  element \(x \in X\) is said to be on \emph{normal form} with 
%  respect to $\rightarrow$ if \(x \nrightarrow y\) for all \(y \in 
%  X\), i.e., if there is no \(y \in X\) such that \(x \rightarrow 
%  y\). To be \emph{irreducible} and to be \emph{terminal} are two 
%  common synonyms of `to be on normal form'.
%  
%  Denote by $\stackrel{*}{\rightarrow}$ the reflexive--transitive 
%  closure of $\rightarrow$. An element \(y \in X\) is said to 
%  \emph{a normal form of} \(x \in X\) if $y$ is on normal form and 
%  \(x \stackrel{*}{\rightarrow} y\).
%\end{definition}
%
%In these terms, condition~\ref{Cond:UniqueNF} of 
%Theorem~\ref{S:NDL} can be more succintly stated as ``every element 
%of $X$ has a unique normal form''. 

The theorem has several immediate 
applications\Ldash first of all that there is a unique function 
\(N\colon X \longrightarrow X\) which sends arbitrary elements of 
$X$ to their normal forms\Dash and this leads to a test for whether 
arbitrary elements are equivalent.

\begin{lem} \label{L:EquivAndNF}
  Let $X$ be a set and $\rightarrow$ a binary relation on $X$ such 
  that all elements of $X$ has a unique normal form with respect to 
  $\rightarrow$. Denote by $\sim$ the reflexive transitive symmetric 
  closure of $\rightarrow$ and denote by $N$ the function 
  \(X \longrightarrow X\) which sends every element to its normal 
  form with respect to $\rightarrow$. Then for all \(x,y \in X\),
  \begin{equation}
     x \sim y \quad\text{if and only if}\quad N(x)=N(y) \text{.}
  \end{equation}
\end{lem}
\begin{proof}
  Denote by $\stackrel{*}{\rightarrow}$ the reflexive--transitive 
  closure of $\rightarrow$ and denote by $\leftrightarrow$ the 
  symmetrisation of $\rightarrow$. If \(N(x)=N(y)\) then obviously 
  \(x \stackrel{*}{\rightarrow} N(x) = N(y) \stackrel{*}{\leftarrow} 
  y\) and thus \(x \sim N(x) \sim y\) as claimed. If conversely \(x 
  \sim y\) then there must exist some sequence
  $$
    z_0 \leftrightarrow z_1 \leftrightarrow \dotsb \leftrightarrow 
    z_{n-1} \leftrightarrow z_n
  $$
  where \(z_0,\dotsc,z_n \in X\), \(z_0=x\), and \(z_n=y\). For every 
  $z_k$ one of \(z_k \rightarrow z_{k+1}\) and \(z_k \leftarrow 
  z_{k+1}\) must hold. In the former case \(z_k 
  \stackrel{*}{\rightarrow} z_{k+1} \stackrel{*}{\rightarrow} 
  N(z_{k+1})\) and thus by transitivity \(z_k \stackrel{*}{\rightarrow} 
  N(z_{k+1})\), but this means $N(z_{k+1})$ is a normal form also 
  of $z_k$. In the latter case one similarly shows that $N(z_k)$ is a 
  normal form of $z_{k+1}$. Either way it follows from the uniqueness 
  of the normal form that \(N(z_k) = N(z_{k+1})\). Thus \(N(z_0) = 
  N(z_1) = \dotsb = N(z_{n-1}) = N(z_n)\) and hence \(N(x)=N(y)\) as 
  claimed.
\end{proof}


Another application of normal forms is to serve as representatives of 
the equivalence classes of $X$. Many mathematical constructions of 
algebraic structures (e.g.~that of a quotient of a group) end up  
producing a set $X/{\sim}$ of equivalence classes of some given set 
$X$, but these are as a rule unfeasible for computational purposes. 
What one can do instead is to choose an element from each equivalence 
class and use these as representatives of their equivalence classes. 
When normal forms are known to be unique, there trivially exists a 
bijection from
$$
  X_{\mathrm{nf}} = \setOf{ x \in X }{ \text{$x$ is on normal form} }
$$
to $X/{\sim}$ given by \(x \mapsto [x]_\sim\), and its inverse can 
thanks to Lemma~\ref{L:EquivAndNF} be defined in terms of the normal 
form map $N$ as \([x]_\sim \mapsto N(x)\).

How does one know that the normal form map $N$ is effective, though? 
This is because of the terminating property on the binary relation 
in Theorem~\ref{S:NDL}.

%\begin{definition}
%  Let $X$ be a set. A strict binary relation $\rightarrow$ on $X$ 
%  is said to satisfy the \emph{descending chain condition} (or be 
%  \emph{DCC} for short) if no infinite sequence 
%  \(\{x_n\}_{n=1}^\infty \subseteq X\) satisfies 
%  \(x_n \rightarrow x_{n+1}\) for all \(n \geqslant 1\).
%\end{definition}
%
%In other words, $\rightarrow$ satisfies the descending chain 
%condition if every (strictly) $\rightarrow$-descending sequence
%$$
%  x_1 \rightarrow x_2 \rightarrow x_3 \rightarrow \dotsb
%$$
%terminates after a finite number of steps. The business about 
%strictness is because this is a definition that is suitable for 
%strict inequality relations such as $>$. If one instead wishes to 
%state the condition in terms of its non-strict counterpart 
%$\geqslant$, one would rather say that $\geqslant$ satisfies the 
%descending chain condition if every infinite $\geqslant$-descending 
%sequence
%$$
%  x_1 \geqslant x_2 \geqslant x_3 \geqslant \dotsb
%$$
%is eventually constant, i.e., \(x_n = x_{n+1} = x_{n+2} = \dotsb\) 
%for some natural number $n$. One consequence of $\rightarrow$ being 
%DCC is that there cannot be any closed loops \(x_1 \rightarrow x_2 
%\rightarrow \dotsb \rightarrow x_n \rightarrow x_1\), and from this 
%follows that the reflexive--transitive closure 
%$\stackrel{*}{\rightarrow}$ is a partial order (rather than a 
%quasiorder). 

The primary consequence of being terminating is that 
$\stackrel{*}{\rightarrow}$ is well-founded, and it is no surprise that Theorem~\ref{S:NDL} is proved using well-founded induction. Being terminating is also what guarantees that the following algorithm ``terminates''.

\begin{alg}[Normal form] \label{Alg:NormalForm}
  Let $X$ be a set and $\rightarrow$ a strict binary relation on $X$ 
  which satisfies the descending chain condition. 
  \begin{description}
    \item[Input]
      An element \(x \in X\).
    \item[Output]
      A normal form of $x$.
    \item[Procedure]
      If $x$ is on normal form with respect to $\rightarrow$ then 
      return $x$. Otherwise there is some \(y \in X\) such that \(x 
      \rightarrow y\); pick one such $y$. Set \(x:=y\) and repeat 
      the procedure from start.
  \end{description}
\end{alg}

For \PMlinkescapetext{relations} $\rightarrow$ which 
\PMlinkescapetext{satisfy} the descending chain 
condition and have unique normal forms, Algorithm~\ref{Alg:NormalForm} 
and Lemma~\ref{L:EquivAndNF} 
combine to an algorithm for deciding the 
relation $\sim$. A typical problem is however that uniqueness of the 
normal form is a global condition that is very hard to establish 
using elementary methods. 
\emph{Theorem~\ref{S:NDL} offers an 
equivalent local condition that often is straightforward to verify in 
each particular case by explicit calculations.}


\section{Diamond lemmas for algebraic structures}

One disadvantage of the basic diamond lemma (Theorem~\ref{S:NDL}) is 
that it requires all reductions to be explicit, whereas the 
\PMlinkescapetext{average} mathematician probably expects it 
to handle reduction schemas transparently. To illustate the 
\PMlinkescapetext{difference}, consider again the 
reduction \(a(b +\nobreak c) \rightarrow ab+ac\). This does 
\emph{not} imply \((d +\nobreak e)(b +\nobreak c) \rightarrow 
(d +\nobreak e)b + (d +\nobreak e)c\), or even \(2(b +\nobreak c) 
\rightarrow 2b + 2c\), because all three left hand 
\PMlinkescapetext{sides} are distinct as formulae 
(i.e., as elements of $X$). In \PMlinkescapetext{order} to let $a$, $b$, and 
$c$ be arbitrary, one need to explicitly state that e.g.\@ 
``\(a(b +\nobreak c) \rightarrow ab+ac\) for all well-formed formulae 
$a$, $b$, and $c$''. Even this may be less than what was intended, 
since it is often the case that one wants it to follow from 
\(a(b +\nobreak c) \rightarrow ab+ac\) also that 
\(f\bigl( a(b +\nobreak c) \bigr) \rightarrow f(ab +\nobreak ac)\) 
for arbitrary functionalised expressions $f$. Theorem~\ref{S:NDL} 
requires its user to take care of such details explicitly.

Moreover, the conditions in Theorem~\ref{S:NDL} are stated about 
arbitrary elements of $X$, so even if one employs schemas in the 
definition of $\rightarrow$, it is necessary to verify that 
condition~\ref{Cond:Diamond} holds for every concrete triple $(a,b,c)$ 
of elements of $X$ such that \(b \leftarrow a \rightarrow c\). This too 
can be done collectively using proof schemas, but the situation is 
complicated enough that it is often far from clear whether all cases 
have been checked. It is furthermore common that trivial 
verifications of a syntactic origin outnumber the verifications that 
have interesting mathematical content. This makes it less likely that 
mathematicians will actually bother to produce a 
\PMlinkescapetext{complete} proof, and 
more likely that they will skip it altogether by resorting to the 
notorious ``It is easy to see that \dots''

Finally, it is not always the case than one just wants an equivalence 
relation. When the \PMlinkescapetext{base} set $X$ has an 
established \PMlinkescapetext{algebraic} structure one typically 
rather wants $\sim$ to be a congruence 
relation, but the basic diamond lemma can of course not take such 
sophistications into account. Therefore \emph{there exist also a number 
of specialised diamond lemmas which are tailored to particular 
\PMlinkescapetext{algebraic structures}} 
and can make do with verifications of significantly fewer 
cases than the basic diamond lemma.

\medskip

The most familiar such result is probably the \emph{diamond lemma for 
ring theory} of Bergman~\cite{Bergman}, also known as the 
\emph{composition lemma for associative algebras}~\cite{Bokut}.

\bigskip

[Give full statement of theorem? It's quite a mouthful.]

[Write about relation to Gr\"obner basis theory.]



\begin{thebibliography}{9}
%
\bibitem{Bergman} 
   \textsc{G. M. Bergman}: 
   The Diamond Lemma for Ring Theory, 
   \textit{Adv.\ Math.} \textbf{29} (1978), 178--218.
%
\bibitem{Bokut}
   \textsc{L. A. Bokut\cprime}:
   Embeddings into simple associative algebras (Russian),
   \textit{Algebra i Logika} \textbf{15}, no.~2 (1976), 
   pp.~117--142 and~245.
   English translation in \textit{Algebra and Logic}, pp.~73--90.
%
\bibitem{Newman} 
   \textsc{M. H. A. Newman}: 
   On theories with a combinatorial definition of ``equivalence'', 
   \textit{Ann. of Math.} \textbf{43} (1942), 223--243.
%
\end{thebibliography}
%%%%%
%%%%%
\end{document}
