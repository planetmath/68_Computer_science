\documentclass[12pt]{article}
\usepackage{pmmeta}
\pmcanonicalname{Confluence}
\pmcreated{2013-03-22 17:47:24}
\pmmodified{2013-03-22 17:47:24}
\pmowner{CWoo}{3771}
\pmmodifier{CWoo}{3771}
\pmtitle{confluence}
\pmrecord{14}{40251}
\pmprivacy{1}
\pmauthor{CWoo}{3771}
\pmtype{Definition}
\pmcomment{trigger rebuild}
\pmclassification{msc}{68Q42}
\pmrelated{ChurchRosserProperty}
\pmrelated{AmalgamationProperty}
\pmrelated{NormalizingReduction}
\pmrelated{TerminatingReduction}
\pmdefines{confluent}
\pmdefines{joinable}
\pmdefines{semi-confluent}

\usepackage{amssymb,amscd}
\usepackage{amsmath}
\usepackage{amsfonts}
\usepackage{mathrsfs}

% used for TeXing text within eps files
%\usepackage{psfrag}
% need this for including graphics (\includegraphics)
%\usepackage{graphicx}
% for neatly defining theorems and propositions
\usepackage{amsthm}
% making logically defined graphics
%%\usepackage{xypic}
\usepackage{pst-plot}

% define commands here
\newcommand*{\abs}[1]{\left\lvert #1\right\rvert}
\newtheorem{prop}{Proposition}
\newtheorem{thm}{Theorem}
\newtheorem{ex}{Example}
\newcommand{\real}{\mathbb{R}}
\newcommand{\pdiff}[2]{\frac{\partial #1}{\partial #2}}
\newcommand{\mpdiff}[3]{\frac{\partial^#1 #2}{\partial #3^#1}}
\begin{document}
Call a binary relation $\to$ on a set $S$ a reduction.  Let $\to^*$ be the reflexive transitive closure of $\to$.  Two elements $a,b\in S$ are said to be \emph{joinable} if there is an element $c\in S$ such that $a\to^* c$ and $b\to^* c$.  Graphically, this means that 
\begin{center}
$
\xymatrix@R-=10pt{
a\ar[dr]^{*}\\
&c\\
b\ar[ur]_{*}
}
$
\end{center}
where the starred arrows represent 
$$a\to a_1\to \cdots \to a_n \to c\qquad\mbox{ and }\qquad b\to b_1\to \cdots \to b_m \to c$$
respectively ($m,n$ are non-negative integers).  The case $m=0$ (or $n=0$) means $a\to c$ (or $b\to c$).

\textbf{Definition}.  $\to$ is said to be \emph{confluent} if whenever $x\to^* a$ and $x\to^*b$, then $a,b$ are joinable.   In other words, $\to$ is confluent iff $\to^*$ has the amalgamation property.  Graphically, this says that, whenever we have a diagram

\begin{center}
$
\xymatrix@R-=10pt{
&a\\
x\ar[ur]^{*}\ar[dr]_{*} &\\
&b
}
$
\end{center}
then it can be ``completed'' into a ``diamond'':
\begin{center}
$
\xymatrix@R-=10pt{
&a\ar[dr]^{*}\\
x\ar[ur]^{*}\ar[dr]_{*} &&c\\
&b\ar[ur]_{*}&
}
$
\end{center}

\textbf{Remark}.  A more general property than confluence, called \emph{semi-confluence} is defined as follows: $\to$ is \emph{semi-confluent} if for any triple $x,a,b\in S$ such that $x\to a$ and $x\to^* b$, then $a,b$ are joinable.  It turns out that this seemingly weaker notion is actually equivalent to the stronger notion of confluence.  In addition, it can be shown that $\to$ is confluent iff $\to$ has the Church-Rosser property.

\begin{thebibliography}{8}
\bibitem{bn} F. Baader, T. Nipkow, \emph{Term Rewriting and All That}, Cambridge University Press (1998).
\end{thebibliography}
%%%%%
%%%%%
\end{document}
