\documentclass[12pt]{article}
\usepackage{pmmeta}
\pmcanonicalname{WeakBisimulation}
\pmcreated{2013-03-22 19:30:55}
\pmmodified{2013-03-22 19:30:55}
\pmowner{CWoo}{3771}
\pmmodifier{CWoo}{3771}
\pmtitle{weak bisimulation}
\pmrecord{10}{42490}
\pmprivacy{1}
\pmauthor{CWoo}{3771}
\pmtype{Definition}
\pmcomment{trigger rebuild}
\pmclassification{msc}{68Q85}
\pmsynonym{invisible step}{WeakBisimulation}
\pmdefines{silent step}
\pmdefines{weak simulation}

\endmetadata

\usepackage{amssymb,amscd}
\usepackage{amsmath}
\usepackage{amsfonts}
\usepackage{mathrsfs}

% used for TeXing text within eps files
%\usepackage{psfrag}
% need this for including graphics (\includegraphics)
%\usepackage{graphicx}
% for neatly defining theorems and propositions
\usepackage{amsthm}
% making logically defined graphics
%%\usepackage{xypic}
\usepackage{pst-plot}

% define commands here
\newcommand*{\abs}[1]{\left\lvert #1\right\rvert}
\newtheorem{prop}{Proposition}
\newtheorem{thm}{Theorem}
\newtheorem{ex}{Example}
\newcommand{\real}{\mathbb{R}}
\newcommand{\pdiff}[2]{\frac{\partial #1}{\partial #2}}
\newcommand{\mpdiff}[3]{\frac{\partial^#1 #2}{\partial #3^#1}}

\begin{document}
Let $M=(S,\Sigma,\rightarrow)$ be a labelled state transition system (LTS).  Recall that for each label $\alpha \in \Sigma$, there is an associated binary relation $\stackrel{\alpha}{\rightarrow}$ on $S$.  Single out a label $\tau\in \Sigma$, and call it the \emph{silent step}.  Define the following relations:
\begin{enumerate}
\item Let $\Rightarrow$ be the reflexive and transitive closures of $\stackrel{\tau}{\rightarrow}$.  In other words, $p \Rightarrow q$ iff either $p=q$, or there is a positive integer $n>1$ and states $r_1,\ldots, r_n$ such that $p= r_1$ and $q=r_n$ and $r_i \stackrel{\tau}{\rightarrow}  r_{i+1}$, where $i=1,\ldots, n-1$.
\item Next, for any label $\alpha$ that is not the silent step $\tau$ in $\Sigma$, define $$\stackrel{\alpha}{\Rightarrow}\;\;\; := \;\;\; \stackrel{\alpha}{\rightarrow} \circ \Rightarrow \circ \stackrel{\alpha}{\rightarrow},$$
where $\circ$ denotes the relational composition operation.  In other words, $p \stackrel{\alpha}{\Rightarrow} q$ iff there are states $r$ and $s$ such that $p \stackrel{\alpha}{\rightarrow} r$, $r \Rightarrow s$, and $s \stackrel{\alpha}{\rightarrow} q$.
\item Finally, for any label $\alpha\in \Sigma$, let 
\begin{displaymath}
\stackrel{(\alpha)}{\Rightarrow}\;\;\; := \;\;\; \left\{
\begin{array}{ll}
\Rightarrow & \mbox{if }\alpha=\tau\\
\stackrel{\alpha}{\Rightarrow} & \textrm{otherwise.}
\end{array}
\right.
\end{displaymath}
\end{enumerate}

\textbf{Definition}.  Let $M=(S_1,\Sigma,\rightarrow_1)$ and $N=(S_2,\Sigma,\rightarrow_2)$ be two labelled state transition systems, with $\tau \in \Sigma$ the silent step.  A relation $\approx \subseteq S_1 \times S_2$ is called a \emph{weak simulation} if whenever $p \approx q$ and any labelled transition $p \stackrel{\alpha}{\rightarrow}_1 p'$, there is a state $q'\in S_2$ such that $p' \approx q'$ and $p' \stackrel{(\alpha)}{\Rightarrow}_2 q$.  $\approx$ is a \emph{weak bisimulation} if both $\approx$ and its converse $\approx^{-1}$ are weak simulations.

%%%%%
%%%%%
\end{document}
