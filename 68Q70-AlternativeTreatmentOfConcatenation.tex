\documentclass[12pt]{article}
\usepackage{pmmeta}
\pmcanonicalname{AlternativeTreatmentOfConcatenation}
\pmcreated{2013-03-22 17:24:10}
\pmmodified{2013-03-22 17:24:10}
\pmowner{rspuzio}{6075}
\pmmodifier{rspuzio}{6075}
\pmtitle{alternative treatment of concatenation}
\pmrecord{6}{39774}
\pmprivacy{1}
\pmauthor{rspuzio}{6075}
\pmtype{Definition}
\pmcomment{trigger rebuild}
\pmclassification{msc}{68Q70}
\pmclassification{msc}{20M35}
\pmrelated{Word}

% this is the default PlanetMath preamble.  as your knowledge
% of TeX increases, you will probably want to edit this, but
% it should be fine as is for beginners.

% almost certainly you want these
\usepackage{amssymb}
\usepackage{amsmath}
\usepackage{amsfonts}

% used for TeXing text within eps files
%\usepackage{psfrag}
% need this for including graphics (\includegraphics)
%\usepackage{graphicx}
% for neatly defining theorems and propositions
%\usepackage{amsthm}
% making logically defined graphics
%%%\usepackage{xypic}

% there are many more packages, add them here as you need them

% define commands here

\begin{document}
It is possible to define words and concatenation in terms of ordered sets.  Let $A$
be a set, which we shall call our alphabet.  Define a \emph{word} on $A$ to be a map 
from a totally ordered set into $A$.  (In order to have words in the usual sense, the
ordered set should be finite but, as the definition presented here does not require 
this condition, we do not impose it.)

Suppose that we have totally ordered sets $(u,<)$ and $(v,\prec)$ and words $f \colon u \to A$
and $g \colon v \to A$.  Let $u \coprod v$ denote the disjoint union of $u$ and $v$ and let
$p \colon u \to u \coprod v$ and $q \colon u \to u \coprod v$ be the canonical maps.  Then
we may define an order $\ll$ on $u \coprod v$ as follows:
\begin{itemize}
\item If $x \in u$ and $y \in u$, then $p(x) \ll p(y)$ if and only if $x < y$.
\item If $x \in u$ and $y \in v$, then $p(x) \ll q(y)$.
\item If $x \in v$ and $y \in v$, then $q(x) \ll q(y)$ if and only if $x \prec y$.
\end{itemize}
We define the \emph{concatenation} of $f$ and $g$, which will be denoted $f \circ g$, to be
map from $u \coprod v$ to $A$ defined by the following conditions:
\begin{itemize}
\item If $x \in u$, then $(f \circ g) (p(x)) = f(x)$.
\item If $y \in u$, then $(f \circ g) (q(x)) = g(x)$.
\end{itemize}
%%%%%
%%%%%
\end{document}
