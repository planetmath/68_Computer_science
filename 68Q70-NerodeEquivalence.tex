\documentclass[12pt]{article}
\usepackage{pmmeta}
\pmcanonicalname{NerodeEquivalence}
\pmcreated{2013-03-22 18:52:11}
\pmmodified{2013-03-22 18:52:11}
\pmowner{Ziosilvio}{18733}
\pmmodifier{Ziosilvio}{18733}
\pmtitle{Nerode equivalence}
\pmrecord{4}{41711}
\pmprivacy{1}
\pmauthor{Ziosilvio}{18733}
\pmtype{Definition}
\pmcomment{trigger rebuild}
\pmclassification{msc}{68Q70}
\pmclassification{msc}{20M35}
\pmdefines{maximality property of Nerode equivalence}

\endmetadata

% this is the default PlanetMath preamble.  as your knowledge
% of TeX increases, you will probably want to edit this, but
% it should be fine as is for beginners.

% almost certainly you want these
\usepackage{amssymb}
\usepackage{amsmath}
\usepackage{amsfonts}

% used for TeXing text within eps files
%\usepackage{psfrag}
% need this for including graphics (\includegraphics)
%\usepackage{graphicx}
% for neatly defining theorems and propositions
%\usepackage{amsthm}
% making logically defined graphics
%%%\usepackage{xypic}

% there are many more packages, add them here as you need them

% define commands here

\begin{document}
Let $S$ be a semigroup and let $X\subseteq S$.
The relation
\begin{equation} \label{eq:ne}
s_1\mathcal{N}_X s_2
\;\;\iff\;\;
\forall t\in S (s_1t\in X\;\iff\;s_2t\in X)
\end{equation}
is an equivalence relation over $S$,
called the \emph{Nerode equivalence} of $X$.

As an example,
if $S=(\mathbb{N},+)$ and
\begin{math}
X=\{n\in\mathbb{N}\mid\exists k\in\mathbb{N}\mid n=3k\},
\end{math}
then $m\mathcal{N}_X n$ iff $m\mod 3=n\mod 3$.

The Nerode equivalence is right-invariant,
\emph{i.e.}, if $s_1\mathcal{N}_Xs_2$
then $s_1t\mathcal{N}_Xs_2t$ for any $t$.
However, it is usually not a congruence.

The Nerode equivalence is maximal in the following sense:
\begin{itemize}
\item if $\eta$ is a right-invariant equivalence over $S$ %%that saturates $X$,
and $X$ is union of classes of $\eta$,
\item then $s\eta t$ implies $s\mathcal{N}_X t$.
\end{itemize}
In fact, let $r\in S$:
since $s\eta t$ and $\eta$ is right-invariant, $sr\eta tr$.
However, $X$ is union of classes of $\eta$,
therefore $sr$ and $tr$
are either both in $X$ or both outside $X$.
This is true for all $r\in S$, thus $s\mathcal{N}_X t$.

The Nerode equivalence is linked to the syntactic congruence 
by the following fact, whose proof is straightforward:
\begin{displaymath}
s_1\equiv_X s_2
\;\;\mathrm{iff}\;\;
ls_1\mathcal{N}_X ls_2\;\forall l\in S\;.
\end{displaymath}

%%%%%
%%%%%
\end{document}
