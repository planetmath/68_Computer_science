\documentclass[12pt]{article}
\usepackage{pmmeta}
\pmcanonicalname{RandomTuringMachine}
\pmcreated{2013-03-22 13:01:53}
\pmmodified{2013-03-22 13:01:53}
\pmowner{Henry}{455}
\pmmodifier{Henry}{455}
\pmtitle{random Turing machine}
\pmrecord{5}{33430}
\pmprivacy{1}
\pmauthor{Henry}{455}
\pmtype{Definition}
\pmcomment{trigger rebuild}
\pmclassification{msc}{68Q10}
\pmdefines{one sided error}
\pmdefines{two sided error}
\pmdefines{one-sided error}
\pmdefines{two-sided error}
\pmdefines{minimal error}

% this is the default PlanetMath preamble.  as your knowledge
% of TeX increases, you will probably want to edit this, but
% it should be fine as is for beginners.

% almost certainly you want these
\usepackage{amssymb}
\usepackage{amsmath}
\usepackage{amsfonts}

% used for TeXing text within eps files
%\usepackage{psfrag}
% need this for including graphics (\includegraphics)
%\usepackage{graphicx}
% for neatly defining theorems and propositions
%\usepackage{amsthm}
% making logically defined graphics
%%%\usepackage{xypic}

% there are many more packages, add them here as you need them

% define commands here
%\PMlinkescapeword{theory}
\begin{document}
A random Turing machine is defined the same way as a non-deterministic Turing machine, but with different rules governing when it accepts or rejects.  Whenever there are multiple legal moves, instead of always guessing right, a random machine selects one of the possible moves at random.

As with non-deterministic machines, this can also be viewed as a deterministic machine with an extra input, which corresponds to the random selections.

There are several different ways of defining what it means for a random Turing machine to accept or reject an input.  Let $\operatorname{Prob}_T(x)$ be the probability that $T$ halts in an accepting state when the input is $x$.

A positive \emph{one-sided error} machine is said to accept $x$ if $\operatorname{Prob}_T(x)\geq\frac{1}{2}$ and to reject $x$ if $\operatorname{Prob}_T(x)=0$.  A negative one-sided error machine accepts $x$ if $\operatorname{Prob}_T(x)=1$ and rejects $x$ if $\operatorname{Prob}_T(x)\leq\frac{1}{2}$.  So a single run of a positive one-sided error machine never misleadingly accepts but may misleadingly reject, while a single run of a negative one-sided error machine never misleadingly rejects.

The definition of a positive one-sided error machine is stricter than the definition of a non-deterministic machine, since a non-deterministic machine rejects when there is no certificate and accepts when there is at least one, while a positive one-sided error machine requires that half of all possible guess inputs be certificates.

A \emph{two-sided error} machine accepts $x$ if $\operatorname{Prob}_T(x)\geq\frac{2}{3}$ and rejects $x$ if $\operatorname{Prob}_T(x)\leq\frac{1}{3}$.

The constants in any of the definitions above can be adjusted, although this will affect the time and space complexity classes.

A \emph{minimal error} machine accepts $x$ if $\operatorname{Prob}_T(x)>\frac{1}{2}$ and rejects $x$ if $\operatorname{Prob}_T(x)<\frac{1}{2}$.

One additional variant defines, in addition to accepting states, rejecting states.  Such a machine is called \emph{zero error} if on at least half of all inputs it halts in either an accepting or a rejecting state.  It accepts $x$ if there is any sequence of guesses which causes it to end in an accepting state, and rejects if there is any sequence of guesses which causes it to end in a rejecting state.  In other words, such a machine is never wrong when it provides an answer, but does not produce a decisive answer on all input.  The machine can emulate a positive (resp. negative) one-sided error machine by accepting (resp. rejecting) when the result is indecisive.

It is a testament to the robustness of the definition of the Turing machine (and the Church-Turing thesis) that each of these definitions computes the same functions as a standard Turing machine.  The point of defining all these types of machines is that some are more efficient than others, and therefore they define different time and space complexity classes.
%%%%%
%%%%%
\end{document}
