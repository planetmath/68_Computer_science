\documentclass[12pt]{article}
\usepackage{pmmeta}
\pmcanonicalname{BinaryGrayCode}
\pmcreated{2013-03-22 12:30:09}
\pmmodified{2013-03-22 12:30:09}
\pmowner{mathcam}{2727}
\pmmodifier{mathcam}{2727}
\pmtitle{binary Gray code}
\pmrecord{9}{32733}
\pmprivacy{1}
\pmauthor{mathcam}{2727}
\pmtype{Definition}
\pmcomment{trigger rebuild}
\pmclassification{msc}{68P30}
\pmclassification{msc}{05C45}
\pmsynonym{Gray code}{BinaryGrayCode}
\pmdefines{cyclic Gray code}

\endmetadata

% this is the default PlanetMath preamble.  as your knowledge
% of TeX increases, you will probably want to edit this, but
% it should be fine as is for beginners.

% almost certainly you want these
\usepackage{amssymb}
\usepackage{amsmath}
\usepackage{amsfonts}

% used for TeXing text within eps files
%\usepackage{psfrag}
% need this for including graphics (\includegraphics)
%\usepackage{graphicx}
% for neatly defining theorems and propositions
%\usepackage{amsthm}
% making logically defined graphics
%%%\usepackage{xypic} 

% there are many more packages, add them here as you need them

% define commands here
\begin{document}
An $n$-bit binary \emph{Gray code} is a non-repeating sequence of the integers from $0$ to $2^n-1$ inclusive such that the binary representation of each number in the sequence differs by exactly one bit from the binary representation of the previous number: that is, the Hamming distance between consecutive elements is $1$.  In addition, we also define a \emph{cyclic Gray code} to be a Gray code where an extra condition is imposed:  The last number in the sequence must differ by exactly one bit from the first number in the sequence.

For example, one $3$-bit cyclic Gray code is:

$$000_2$$
$$010_2$$
$$011_2$$
$$001_2$$
$$101_2$$
$$111_2$$
$$110_2$$
$$100_2$$

There is a one-to-one correspondence between all possible $n$-bit Gray codes and all possible Hamiltonian cycles on an $n$-dimensional hypercube.  (To see why this is so, imagine assigning a binary number to each vertex of a hypercube where an edge joins each pair of vertices that differ by exactly one bit.)
%%%%%
%%%%%
\end{document}
