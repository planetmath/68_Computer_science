\documentclass[12pt]{article}
\usepackage{pmmeta}
\pmcanonicalname{CharacteristicMonoid}
\pmcreated{2013-03-22 19:01:10}
\pmmodified{2013-03-22 19:01:10}
\pmowner{CWoo}{3771}
\pmmodifier{CWoo}{3771}
\pmtitle{characteristic monoid}
\pmrecord{8}{41890}
\pmprivacy{1}
\pmauthor{CWoo}{3771}
\pmtype{Definition}
\pmcomment{trigger rebuild}
\pmclassification{msc}{68Q70}
\pmclassification{msc}{20M35}
\pmclassification{msc}{03D05}
\pmclassification{msc}{68Q45}

\usepackage{amssymb,amscd}
\usepackage{amsmath}
\usepackage{amsfonts}
\usepackage{mathrsfs}

% used for TeXing text within eps files
%\usepackage{psfrag}
% need this for including graphics (\includegraphics)
%\usepackage{graphicx}
% for neatly defining theorems and propositions
\usepackage{amsthm}
% making logically defined graphics
%%\usepackage{xypic}
\usepackage{pst-plot}

% define commands here
\newcommand*{\abs}[1]{\left\lvert #1\right\rvert}
\newtheorem{prop}{Proposition}
\newtheorem{thm}{Theorem}
\newtheorem{ex}{Example}
\newcommand{\real}{\mathbb{R}}
\newcommand{\pdiff}[2]{\frac{\partial #1}{\partial #2}}
\newcommand{\mpdiff}[3]{\frac{\partial^#1 #2}{\partial #3^#1}}
\begin{document}
Given a semiautomaton $M=(S,\Sigma,\delta)$, the transition function is typically defined as a function from $S\times \Sigma$ to $S$.  One may instead think of $\delta$ as a set $C(M)$ of functions $$C(M):=\lbrace \delta_a: S\to S \mid a \in \Sigma \rbrace, \qquad \mbox{where} \qquad \delta_a(s):=\delta(s,a).$$
Since the transition function $\delta$ in $M$ can be extended to the domain $S\times \Sigma^*$, so can the set $C(M)$: $$C(M):=\lbrace \delta_u: S\to S\mid u\in \Sigma^*\rbrace,\qquad \mbox{where}\qquad \delta_u(s):=\delta(s,u).$$

The advantage of this interpreation is the following: for any input words $u,v$ over $\Sigma$: $$\delta_u\circ \delta_v= \delta_{vu},$$
which can be easily verified:
$$\delta_{vu}(s)=\delta(s,vu)=\delta(\delta(s,v),u) = \delta(\delta_v(s),u)=\delta_u(\delta_v(s))=(\delta_u\circ \delta_v)(s).$$

In particular, $\delta_{\lambda}$ is the identity function on $S$, so that the set $C(M)$ becomes a monoid, called the \emph{characteristic monoid} of $M$.

The characteristic monoid $C(M)$ of a semiautomaton $M$ is related to the free monoid $\Sigma^*$ generated by $\Sigma$ in the following manner: define a binary relation $\sim$ on $\Sigma^*$ by $u\sim v$ iff $\delta_u = \delta_v$.  Then $\sim$ is clearly an equivalence relation on $\Sigma^*$.  It is also a congruence relation with respect to concatenation: if $u\sim v$, then for any $w$ over $\Sigma$:
$$\delta_{uw}(s)=\delta_u(\delta_w(s))=\delta_v(\delta_w(s))=\delta_{vw}(s)$$
and
$$\delta_{wu}(s)=\delta_w(\delta_u(s))=\delta_w(\delta_v(s))=\delta_{wv}(s).$$
Putting the two together, we see that if $x\sim y$, then $ux \sim vx \sim vy$.  We denote $[u]$ the congruence class in $\Sigma^*/\sim$ containing the word $u$.

Now, define a map $\phi: C(M)\to \Sigma^*/\sim$ by setting $\phi(\delta_u)=[u]$.  Then $\phi$ is well-defined.  Furthermore, under $\phi$, it is easy to see that $C(M)$ is anti-isomorphic to $\Sigma^*/\sim$.

\textbf{Remark}.  In order to avoid using anti-isomorphisms, the usual practice is to introduce a multiplication $\cdot$ on $C(M)$ so that $\delta_u \cdot \delta_v:=\delta_v \circ \delta_u$.  Then $C(M)$ under $\cdot$ is isomorphic to $\Sigma^*/\sim$.

Some properties:
\begin{itemize}
\item If $M$ and $N$ are isomorphic semiautomata with identical input alphabet, then $C(M)=C(N)$.
\item If $N$ is a subsemiautomaton of $M$, then $C(N)$ is a homomorphic image of a submonoid of $C(M)$.
\item If $N$ is a homomorphic image of $M$, so is $C(N)$ a homomorphic image of $C(M)$.
\end{itemize}

\begin{thebibliography}{8}
\bibitem{ag} A. Ginzburg, {\em Algebraic Theory of Automata}, Academic Press (1968).
\bibitem{mi} M. Ito, {\em Algebraic Theory of Automata and Languages}, World Scientific, Singapore (2004).
\end{thebibliography}
%%%%%
%%%%%
\end{document}
