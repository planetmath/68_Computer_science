\documentclass[12pt]{article}
\usepackage{pmmeta}
\pmcanonicalname{SyntacticCongruence}
\pmcreated{2013-03-22 18:52:08}
\pmmodified{2013-03-22 18:52:08}
\pmowner{Ziosilvio}{18733}
\pmmodifier{Ziosilvio}{18733}
\pmtitle{syntactic congruence}
\pmrecord{6}{41710}
\pmprivacy{1}
\pmauthor{Ziosilvio}{18733}
\pmtype{Definition}
\pmcomment{trigger rebuild}
\pmclassification{msc}{68Q70}
\pmclassification{msc}{20M35}
\pmdefines{syntactic semigroup}
\pmdefines{syntactic monoid}
\pmdefines{syntactic morphism}
\pmdefines{maximality property of syntactic congruence}

% this is the default PlanetMath preamble.  as your knowledge
% of TeX increases, you will probably want to edit this, but
% it should be fine as is for beginners.

% almost certainly you want these
\usepackage{amssymb}
\usepackage{amsmath}
\usepackage{amsfonts}

% used for TeXing text within eps files
%\usepackage{psfrag}
% need this for including graphics (\includegraphics)
%\usepackage{graphicx}
% for neatly defining theorems and propositions
%\usepackage{amsthm}
% making logically defined graphics
%%%\usepackage{xypic}

% there are many more packages, add them here as you need them

% define commands here

\begin{document}
Let $S$ be a semigroup and let $X\subseteq S$.
The relation
\begin{equation} \label{eq:sc}
s_1\equiv_X s_2
\;\;\mathrm{iff}\;\;
\forall l,r\in S (ls_1r\in X\;\mathrm{iff}\;ls_2r\in X)
\end{equation}
is called the \emph{syntactic congruence} of $X$.
The quotient $S/\equiv_X$ is called the \emph{syntactic semigroup} of $X$,
and the natural morphism $\phi:S\to S/\equiv_X$
is called the \emph{syntactic morphism} of $X$.
If $S$ is a monoid, then $S/\equiv_X$ is also a monoid,
called the \emph{syntactic monoid} of $X$.

As an example,
if $S=(\mathbb{N},+)$ and
\begin{math}
X=\{n\in\mathbb{N}\mid\exists k\in\mathbb{N}\mid n=3k\},
\end{math}
then $m\equiv_X n$ if $m\mod 3=n\mod 3$,
and the syntactic monoid is isomorphic to the cyclic group of order three.

It is straightforward that $\equiv_X$
is an equivalence relation %%that saturates $X$.
and $X$ is union of classes of $\equiv_X$.
To prove that it is a congruence, let $s_1,s_2,t_1,t_2\in S$
satisfy $s_1\equiv_X s_2$ and $t_1\equiv_X t_2$.
Let $l,r\in S$ be arbitrary.
Then $ls_1t_1r\in X$ iff $ls_2t_1r\in X$ because $s_1\equiv_X s_2$,
and $ls_2t_1r\in X$ iff $ls_2t_2r\in X$ because $t_1\equiv_X t_2$.
Then $s_1t_1\equiv_X s_2t_2$ since $l$ and $r$ are arbitrary.

The syntactic congruence is both left- and right-invariant,
\emph{i.e.}, if $s_1\equiv_Xs_2$,
then $ts_1\equiv_Xts_2$ and $s_1t\equiv_Xs_2t$ for any $t$.

The syntactic congruence is maximal in the following sense:
\begin{itemize}
\item if $\chi$ is a congruence over $S$ %%that saturates $X$,
and $X$ is union of classes of $\chi$,
\item then $s\chi t$ implies $s\equiv_X t$.
\end{itemize}
In fact, let $l,r\in S$:
since $s\chi t$ and $\chi$ is a congruence, $lsr\chi ltr$.
However, $X$ is union of classes of $\chi$,
therefore $lsr$ and $ltr$
are either both in $X$ or both outside $X$.
This is true for all $l,r\in S$, thus $s\equiv_X t$.

%%%%%
%%%%%
\end{document}
