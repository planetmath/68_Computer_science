\documentclass[12pt]{article}
\usepackage{pmmeta}
\pmcanonicalname{ShuffleOfLanguages}
\pmcreated{2013-03-22 18:56:16}
\pmmodified{2013-03-22 18:56:16}
\pmowner{CWoo}{3771}
\pmmodifier{CWoo}{3771}
\pmtitle{shuffle of languages}
\pmrecord{11}{41792}
\pmprivacy{1}
\pmauthor{CWoo}{3771}
\pmtype{Definition}
\pmcomment{trigger rebuild}
\pmclassification{msc}{68Q70}
\pmclassification{msc}{68Q45}
\pmsynonym{shuffle product}{ShuffleOfLanguages}
\pmsynonym{shuffle-closed}{ShuffleOfLanguages}
\pmsynonym{shuffle-closure}{ShuffleOfLanguages}
\pmrelated{PqShuffle}
\pmrelated{DeletionOperationOnLanguages}
\pmrelated{InsertionOperationOnLanguages}
\pmdefines{shuffle}
\pmdefines{shuffle closed}
\pmdefines{shuffle closure}

\usepackage{amssymb,amscd}
\usepackage{amsmath}
\usepackage{amsfonts}
\usepackage{mathrsfs}

% used for TeXing text within eps files
%\usepackage{psfrag}
% need this for including graphics (\includegraphics)
%\usepackage{graphicx}
% for neatly defining theorems and propositions
\usepackage{amsthm}
% making logically defined graphics
%%\usepackage{xypic}
\usepackage{pst-plot}

% define commands here
\newcommand*{\abs}[1]{\left\lvert #1\right\rvert}
\newtheorem{prop}{Proposition}
\newtheorem{thm}{Theorem}
\newtheorem{ex}{Example}
\newcommand{\real}{\mathbb{R}}
\newcommand{\pdiff}[2]{\frac{\partial #1}{\partial #2}}
\newcommand{\mpdiff}[3]{\frac{\partial^#1 #2}{\partial #3^#1}}
\begin{document}
\PMlinkescapeword{order}

Let $\Sigma$ be an alphabet and $u,v$ words over $\Sigma$.  A \emph{shuffle} $w$ of $u$ and $v$ can be loosely defined as a word that is obtained by first decomposing $u$ and $v$ into individual pieces, and then combining (by concatenation) the pieces to form $w$, in a way that the order of the pieces in each of $u$ and $v$ is preserved.  

More precisely, a \emph{shuffle} of $u$ and $v$ is a word of the form
$$u_1v_1\cdots u_k v_k$$
where $u=u_1\cdots u_n$ and $v=v_1\cdots v_n$.  In other words, a shuffle of $u$ and $v$ is either a $k$-insertion of either $u$ into $v$ or $v$ into $u$, for some positive integer $k$.

The set of all shuffles of $u$ and $v$ is called \emph{the} shuffle of $u$ and $v$, and is denoted by $$u \diamond v.$$  The shuffle operation can be more generally applied to languages.  If $L_1, L_2$ are languages, the shuffle of $L_1$ and $L_2$, denoted by $L_1 \diamond L_2$, is the set of all shuffles of words in $L_1$ and $L_2$.  In short, 
$$L_1\diamond L_2 = \bigcup \lbrace u \diamond v \mid u\in L_1, v\in L_2 \rbrace.$$

Clearly, $u\diamond v = \lbrace u\rbrace \diamond \lbrace v\rbrace$.  We shall also identify $L\diamond u$ and $u \diamond L$ with $L \diamond \lbrace u\rbrace$ and $\lbrace u \rbrace \diamond L$ respectively.

A language $L$ is said to be \emph{shuffle closed} if $L\diamond L\subseteq L$.  Clearly $\Sigma^*$ is shuffle closed, and arbitrary intersections shuffle closed languages are shuffle closed.  Given any language $L$, the smallest shuffle closed containing $L$ is called the \emph{shuffle closure} of $L$, and is denoted by $L^{\diamond}$.

It is easy to see that $\diamond$ is a commutative operation: $u\diamond v = v\diamond u$.  It is also not hard to see that $\diamond$ is associative: $(u\diamond v)\diamond w = u\diamond (v\diamond w)$.

In addition, the shuffle operation has the following recursive property: for any $u,v$ over $\Sigma$, and any $a,b\in \Sigma$:
\begin{enumerate}
\item $u \diamond \lambda = \lbrace u \rbrace$,
\item $\lambda \diamond v = \lbrace v \rbrace$,
\item $ua \diamond vb = (ua \diamond v)\lbrace b\rbrace \cup (u\diamond vb)\lbrace a\rbrace$.
\end{enumerate}

For example, suppose $u=aba$, $v=bab$.  Then
\begin{eqnarray*}
u\diamond v &=& [aba \diamond ba]\lbrace b\rbrace \cup [ab \diamond bab]\lbrace a\rbrace \\
&=& [(aba \diamond b) \cup (ab \diamond ba) ]\lbrace ab\rbrace \cup [(ab \diamond ba) \cup (a \diamond bab) ]\lbrace ba\rbrace \\
&=& (ab \diamond ba)\lbrace ab,ba\rbrace \cup (aba\diamond b)\lbrace ab\rbrace \cup (a\diamond bab)\lbrace ba\rbrace \\
&=& \lbrace abba, baab, abab, baba\rbrace \lbrace ab,ba\rbrace \cup \lbrace baba, abba, abab \rbrace \lbrace ab\rbrace \cup \lbrace abab, baab, baba \rbrace \lbrace ba \rbrace \\
&=& \lbrace abba, baab, abab, baba\rbrace \lbrace ab,ba\rbrace \\
&=& \lbrace abbaab, baabab, ababab, babaab, abbaba, baabba, ababba, bababa \rbrace
\end{eqnarray*}

\textbf{Remark}.  Some closure properties with respect to the shuffle operation: let $\mathscr{R}$ be the family of regular languages, and $\mathscr{F}$ the family of context-free languages.  Then $\mathscr{R}$ is closed under $\diamond$.  $\mathscr{F}$ is not closed under $\diamond$.  However, if $L_1\in \mathscr{R}$ and $L_2\in \mathscr{F}$, then $L_1 \diamond L_2 \in \mathscr{F}$.  The proofs of these closure properties can be found in the reference.

\begin{thebibliography}{9}
\bibitem{mi} M. Ito, {\em Algebraic Theory of Automata and Languages}, World Scientific, Singapore (2004).
\end{thebibliography}
%%%%%
%%%%%
\end{document}
