\documentclass[12pt]{article}
\usepackage{pmmeta}
\pmcanonicalname{RegularLanguage}
\pmcreated{2013-03-22 12:26:31}
\pmmodified{2013-03-22 12:26:31}
\pmowner{mps}{409}
\pmmodifier{mps}{409}
\pmtitle{regular language}
\pmrecord{21}{32548}
\pmprivacy{1}
\pmauthor{mps}{409}
\pmtype{Definition}
\pmcomment{trigger rebuild}
\pmclassification{msc}{68Q45}
\pmclassification{msc}{68Q42}
\pmsynonym{type-3 language}{RegularLanguage}
\pmsynonym{type-3 grammar}{RegularLanguage}
\pmsynonym{regular set}{RegularLanguage}
\pmsynonym{regular event}{RegularLanguage}
%\pmkeywords{regular expression}
%\pmkeywords{Chomsky}
\pmrelated{Language}
\pmrelated{DeterministicFiniteAutomaton}
\pmrelated{NonDeterministicFiniteAutomaton}
\pmrelated{RegularExpression}
\pmrelated{KleeneAlgebra}
\pmrelated{ContextFreeLanguage}
\pmrelated{KleenesTheorem}
\pmdefines{regular grammar}

\usepackage{amssymb}
\usepackage{amsmath}
\usepackage{amsfonts}
\begin{document}
A \emph{regular grammar} is a context-free grammar where a production has one of the following three forms:
$$A\to \lambda,\qquad A\to u, \qquad A\to vB$$
where $A,B$ are non-terminal symbols, $u,v$ are terminal words, and $\lambda$ the empty word.  In BNF, they are:
\begin{eqnarray*}
\verb.<.\textit{non-terminal}\verb.>. & ::= & terminal \\
\verb.<.\textit{non-terminal}\verb.>. & ::= & terminal\,\verb.<.\textit{non-terminal}\verb.>. \\
\verb.<.\textit{non-terminal}\verb.>. & ::= & \lambda
\end{eqnarray*}
A \emph{regular language} (also known as a \emph{regular set} or a \emph{regular event}) is the set of strings generated by a regular grammar.
Regular grammars are also known as Type-3 grammars in the Chomsky hierarchy.

A regular grammar can be represented by a deterministic or non-deterministic finite automaton.  Such automata can serve to either generate or accept sentences in a particular regular language.  Note that since the set of regular languages is a subset of context-free languages, any deterministic or non-deterministic finite automaton can be simulated by a pushdown automaton.

There is also a close relationship between regular languages and regular expressions.  With every regular expression we can associate a regular language.  Conversely, every regular language can be obtained from a regular expression.  For example, over the alphabet $\lbrace a,b,c\rbrace$, the regular language associated with the regular expression $a(b\cup c)^*a$ is the set $$\lbrace a\rbrace \circ \lbrace b,c\rbrace^* \circ \lbrace a\rbrace=\lbrace awa\mid w\mbox{ is a word in two letters }b\mbox{ and }c\rbrace,$$ where $\circ$ is the concatenation operation, and $*$ is the Kleene star operation.  Note that $w$ could be the empty word $\lambda$.

Yet another way of describing a regular language is as follows: take any alphabet $\Sigma$.  Let $\mathcal{R}(\Sigma)$ be the smallest subset of $P(\Sigma^*)$ (the power set of the set of words over $\Sigma$, in other words, the set of languages over $\Sigma$), among all subsets of $P(\Sigma^*)$ with the following properties:
\begin{itemize}
\item $\mathcal{R}(\Sigma)$ contains all sets of cardinality no more than 1 (i.e., $\varnothing$ and singletons);
\item $\mathcal{R}(\Sigma)$ is closed under set-theoretic union, concatenation, and Kleene star operations.  
\end{itemize}
Then $L$ is a regular language over $\Sigma$ iff $L\in \mathcal{R}(\Sigma)$.

\textbf{Normal form}.  Every regular language can be generated by a grammar whose productions are either of the form $A\to aB$ or of the form $A\to \lambda$, where $A,B$ are non-terminal symbols, and $a$ is a terminal symbol.  Furthermore, for every pair $(A,a)$, there is exactly one production of the form $A\to aB$.

\textbf{Remark}.  Closure properties on the family of regular languages are: union, intersection, complementation, set difference, concatenation, Kleene star, homomorphism, inverse homomorphism, and reversal.

\begin{thebibliography}{9}
\bibitem{as} A. Salomaa {\em Computation and Automata, Encyclopedia of Mathematics and Its Applications, Vol. 25}. Cambridge (1985).
\end{thebibliography}
%%%%%
%%%%%
\end{document}
