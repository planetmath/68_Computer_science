\documentclass[12pt]{article}
\usepackage{pmmeta}
\pmcanonicalname{ChomskyNormalForm}
\pmcreated{2013-03-22 16:21:13}
\pmmodified{2013-03-22 16:21:13}
\pmowner{rspuzio}{6075}
\pmmodifier{rspuzio}{6075}
\pmtitle{Chomsky normal form}
\pmrecord{7}{38488}
\pmprivacy{1}
\pmauthor{rspuzio}{6075}
\pmtype{Definition}
\pmcomment{trigger rebuild}
\pmclassification{msc}{68Q42}
\pmclassification{msc}{68Q45}
\pmrelated{GreibachNormalForm}
\pmrelated{KurodaNormalForm}

% this is the default PlanetMath preamble.  as your knowledge
% of TeX increases, you will probably want to edit this, but
% it should be fine as is for beginners.

% almost certainly you want these
\usepackage{amssymb}
\usepackage{amsmath}
\usepackage{amsfonts}

% used for TeXing text within eps files
%\usepackage{psfrag}
% need this for including graphics (\includegraphics)
%\usepackage{graphicx}
% for neatly defining theorems and propositions
%\usepackage{amsthm}
% making logically defined graphics
%%%\usepackage{xypic}

% there are many more packages, add them here as you need them

% define commands here

\begin{document}
A grammar is said to be of \emph{Chomsky normal form} if every production
has either of the two forms
 \[A \to BC \qquad \mbox{ or } \qquad A \to a \]
where $A,B,C$ are non-terminal symbols, and $a$ is a terminal symbol.

Grammars of this sort are context-free, hence they describe context-free
languages.  Moreover, given any context-free language not containing the empty word $\lambda$, there exists a
Chomsky normal form grammar which describes it.
%%%%%
%%%%%
\end{document}
