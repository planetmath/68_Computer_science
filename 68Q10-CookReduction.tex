\documentclass[12pt]{article}
\usepackage{pmmeta}
\pmcanonicalname{CookReduction}
\pmcreated{2013-03-22 13:01:41}
\pmmodified{2013-03-22 13:01:41}
\pmowner{Henry}{455}
\pmmodifier{Henry}{455}
\pmtitle{Cook reduction}
\pmrecord{7}{33426}
\pmprivacy{1}
\pmauthor{Henry}{455}
\pmtype{Definition}
\pmcomment{trigger rebuild}
\pmclassification{msc}{68Q10}
\pmclassification{msc}{68Q05}
\pmdefines{Karp reduction}

\endmetadata

% this is the default PlanetMath preamble.  as your knowledge
% of TeX increases, you will probably want to edit this, but
% it should be fine as is for beginners.

% almost certainly you want these
\usepackage{amssymb}
\usepackage{amsmath}
\usepackage{amsfonts}

% used for TeXing text within eps files
%\usepackage{psfrag}
% need this for including graphics (\includegraphics)
%\usepackage{graphicx}
% for neatly defining theorems and propositions
%\usepackage{amsthm}
% making logically defined graphics
%%%\usepackage{xypic}

% there are many more packages, add them here as you need them

% define commands here
%\PMlinkescapeword{theory}
\begin{document}
Given two (search or decision) problems $\pi_1$ and $\pi_2$ and a complexity class $\mathcal{C}$, a $\mathcal{C}$ \emph{Cook reduction} of $\pi_1$ to $\pi_2$ is a Turing machine appropriate for $\mathcal{C}$ which solves $\pi_1$ using $\pi_2$ as an oracle (the Cook reduction itself is not in $\mathcal{C}$, since it is a Turing machine, not a problem, but it should be the class of bounded Turing machines corresponding to $\mathcal{C}$).  The most common type are $\mathcal{P}$ Cook reductions, which are often just called Cook reductions.

If a Cook reduction exists then $\pi_2$ is in some sense ``at least as hard'' as $\pi_1$, since a machine which solves $\pi_2$ could be used to construct one which solves $\pi_1$.  When $\mathcal{C}$ is closed under appropriate operations, if $\pi_2\in\mathcal{C}$ and $\pi_1$ is $\mathcal{C}$-Cook reducible to $\pi_2$ then $\pi_1\in\mathcal{C}$.

A $\mathcal{C}$ \emph{Karp reduction} is a special kind of $\mathcal{C}$ Cook reduction for decision problems $L_1$ and $L_2$.  It is a function $g\in\mathcal{C}$ such that:
$$x\in L_1\leftrightarrow g(x)\in L_2$$

Again, $\mathcal{P}$ Karp reductions are just called Karp reductions.

A Karp reduction provides a Cook reduction, since a Turing machine could decide $L_1$ by calculating $g(x)$ on any input and determining whether $g(x)\in L_2$.  Note that it is a stronger condition than a Cook reduction.  For instance, this machine requires only one use of the oracle.
%%%%%
%%%%%
\end{document}
