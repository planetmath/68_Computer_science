\documentclass[12pt]{article}
\usepackage{pmmeta}
\pmcanonicalname{PumpingLemmaregularLanguages}
\pmcreated{2013-03-22 16:21:17}
\pmmodified{2013-03-22 16:21:17}
\pmowner{rspuzio}{6075}
\pmmodifier{rspuzio}{6075}
\pmtitle{pumping lemma (regular languages)}
\pmrecord{9}{38489}
\pmprivacy{1}
\pmauthor{rspuzio}{6075}
\pmtype{Theorem}
\pmcomment{trigger rebuild}
\pmclassification{msc}{68Q42}
\pmsynonym{pumping lemma}{PumpingLemmaregularLanguages}
\pmrelated{PumpingLemma}

\endmetadata

% this is the default PlanetMath preamble.  as your knowledge
% of TeX increases, you will probably want to edit this, but
% it should be fine as is for beginners.

% almost certainly you want these
\usepackage{amssymb}
\usepackage{amsmath}
\usepackage{amsfonts}

% used for TeXing text within eps files
%\usepackage{psfrag}
% need this for including graphics (\includegraphics)
%\usepackage{graphicx}
% for neatly defining theorems and propositions
\usepackage{amsthm}
% making logically defined graphics
%%%\usepackage{xypic}

% there are many more packages, add them here as you need them

% define commands here
\newtheorem{lemma}{Lemma}
\begin{document}
\begin{lemma}
Let $L$ be a regular language (a.k.a. type 3 language). Then there exist
an integer $n$ such that, if the length of a word $W$ is greater
than $n$, then $W = ABC$ where $A,B,C$ are subwords such that
\begin{enumerate}
\item The length of the subword $B$ is less than $n$.
\item The subword $B$ cannot be empty (although one of $A$ or $C$ might
happen to be empty).
\item For all integers $k > 0$, it is the case that $AB^kC$ belongs to $L$,
where exponentiation denotes repetition of a subword $k$ times.
\end{enumerate}
\end{lemma}

An important use of this lemma is to show that a language
is not regular. (Remember, just because a language happens to be described
in terms of an irregular grammar does not automatically preclude the
possibility of describing the same language also by a
regular grammar.) The idea is to assume that the language is
regular, then arrive at a contradiction via this lemma.

An example of such a use of this lemma is afforded by the language
 \[ L = \{0^p 1^q 0^p \mid p,q > 0 \}.\]
Let $n$ be the number whose existence is guaranteed by the lemma.  
Now, consider the word $W = 0^{n+1} 1^{n+1} 0^{n+1}$.  There must 
exist subwords $A,B,C$ such that $W = ABC$ and $B$ must be of length less than $n$.  The only possibilities are the following
\begin{enumerate}
\item $A = 0^u, B = 0^v, C = 0^{n+1-u-v} 1^{n+1} 0^{n+1}$
\item $A = 0^{n+1-u}, B = 0^u 1^v, C = 1^{n+1-v} 0^{n+1}$
\item $A = 0^{n+1} 1^v, B = 1^u, C = 1^{n+1-u-v} 0^{n+1}$
\item $A = 0^{n+1} 1^{n+1-v}, B = 1^v 0^u, C = 0^{n+1-u}$
\item $A = 0^{n+1} 1^{n+1} 0^u, B = 0^v, C = 0^{n+1-u-v}$
\end{enumerate}
In these cases, $AB^2C$ would have the following form:
\begin{enumerate}
\item $AB^2C = 0^{n+1+v} 1^{n+1} 0^{n+1}$
\item $AB^2C = 0^{n+1} 1^v 0^u 1^{n+1} 0^{n+1}$
\item $AB^2C = 0^{n+1} 1^{n+1+u} 0^{n+1}$
\item $AB^2C = 0^{n+1} 1^{n+1} 0^u 1^v 0^{n+1}$
\item $AB^2C = 0^{n+1} 1^{n+1} 0^{n+1+u}$
\end{enumerate}

It is easy to see that, in each of these five cases, $AB^2C \notin L$.
Hence $L$ cannot be a regular language.


%%%%%
%%%%%
\end{document}
