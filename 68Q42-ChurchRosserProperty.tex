\documentclass[12pt]{article}
\usepackage{pmmeta}
\pmcanonicalname{ChurchRosserProperty}
\pmcreated{2013-03-22 17:47:28}
\pmmodified{2013-03-22 17:47:28}
\pmowner{CWoo}{3771}
\pmmodifier{CWoo}{3771}
\pmtitle{Church-Rosser property}
\pmrecord{9}{40252}
\pmprivacy{1}
\pmauthor{CWoo}{3771}
\pmtype{Definition}
\pmcomment{trigger rebuild}
\pmclassification{msc}{68Q42}
\pmrelated{Confluence}

\usepackage{amssymb,amscd}
\usepackage{amsmath}
\usepackage{amsfonts}
\usepackage{mathrsfs}

% used for TeXing text within eps files
%\usepackage{psfrag}
% need this for including graphics (\includegraphics)
%\usepackage{graphicx}
% for neatly defining theorems and propositions
\usepackage{amsthm}
% making logically defined graphics
%%\usepackage{xypic}
\usepackage{pst-plot}

% define commands here
\newcommand*{\abs}[1]{\left\lvert #1\right\rvert}
\newtheorem{prop}{Proposition}
\newtheorem{thm}{Theorem}
\newtheorem{ex}{Example}
\newcommand{\real}{\mathbb{R}}
\newcommand{\pdiff}[2]{\frac{\partial #1}{\partial #2}}
\newcommand{\mpdiff}[3]{\frac{\partial^#1 #2}{\partial #3^#1}}
\begin{document}
Let $\to$ be a reduction (a binary relation) on a set $S$, and let $\leftrightarrow^*$ be the reflexive transitive symmetric closure of $\to$.  The reduction $\to$ is said to have the \emph{Church-Rosser property} provided that $a\leftrightarrow^* b$ implies that $a$ and $b$ are joinable, for any $a,b\in S$.

In terms of diagrams, the Church-Rosser property means the following, for any $a,b\in S$, if
$$a\leftrightarrow x_1 \leftrightarrow x_2 \leftrightarrow \cdots \leftrightarrow x_n \leftrightarrow b$$ 
where $u\leftrightarrow v$ means $u\to v$ or $u\leftarrow v$ ($:=v\to u$), then there is some $x\in S$ such that 
$$a \to a_1 \cdots \to a_p \to x \leftarrow b_q \leftarrow \cdots \leftarrow b_1 \leftarrow b.$$

\textbf{Remark}.  It can be shown that $\to$ has the Church-Rosser property iff it is confluent.

\begin{thebibliography}{8}
\bibitem{bn} F. Baader, T. Nipkow, \emph{Term Rewriting and All That}, Cambridge University Press (1998).
\end{thebibliography}
%%%%%
%%%%%
\end{document}
