\documentclass[12pt]{article}
\usepackage{pmmeta}
\pmcanonicalname{CognitiveScience}
\pmcreated{2013-03-22 15:31:48}
\pmmodified{2013-03-22 15:31:48}
\pmowner{gprasad}{11015}
\pmmodifier{gprasad}{11015}
\pmtitle{cognitive science}
\pmrecord{6}{37413}
\pmprivacy{1}
\pmauthor{gprasad}{11015}
\pmtype{Topic}
\pmcomment{trigger rebuild}
\pmclassification{msc}{68P20}

% this is the default PlanetMath preamble.  as your knowledge
% of TeX increases, you will probably want to edit this, but
% it should be fine as is for beginners.

% almost certainly you want these
\usepackage{amssymb}
\usepackage{amsmath}
\usepackage{amsfonts}

% used for TeXing text within eps files
%\usepackage{psfrag}
% need this for including graphics (\includegraphics)
%\usepackage{graphicx}
% for neatly defining theorems and propositions
%\usepackage{amsthm}
% making logically defined graphics
%%%\usepackage{xypic}

% there are many more packages, add them here as you need them

% define commands here
\begin{document}
The field of cognitive science consists of an interdisciplinary study of the structures of the human mind. These structures include our sensory/perceptual apparatus, such as vision, audition, olfaction; internal mental processes such as language, thinking, reasoning and problem solving; motor control and the organization of skilled behavior such as speech and musical performance; memory; consciousness; attention; and many other aspects of mind. All of these subfields are clearly intertwined. Disciplines included are psychology, biology, neuroscience, philosophy, anthropology, linguistics, sociology, and computer science; more recently, the academic music world has devoted some of its resources to the study of the cognitive science of music.

Often the claim is made that cognition is information processing: "Cognitive science is the study of information processing, and insofar as a discipline studies that, then it is part of cognitive science." (Hardcastle 1996: 8) This claim is true in the most general sense, in that the mind is continually taking in information and dealing with it in some way. In the past, the further claim was made that the mind is merely an example of a formal symbol-manipulating device that executes mental programs, where any such device would do (e.g. von Neumann 1951). As such, it was assumed that one could ignore the physical and biochemical particulars of the brain. Theories of mental processes then consisted of abstract mathematical models that had the correct input-output characteristics and various functional or causal relations. This kind of theorizing suggested a fundamental dichotomy that was assumed between the body and the mind -- a problematic dichotomy that dates back to Plato.

Overall, the field of cognitive science consists of research on parallel fronts: to name a few, we have the neuroscientific study of the structure of the brain, the psychological study of mental abstractions, and the socioanthropological study of the shaping of these mental structures by culture.
%%%%%
%%%%%
\end{document}
