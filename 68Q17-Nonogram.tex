\documentclass[12pt]{article}
\usepackage{pmmeta}
\pmcanonicalname{Nonogram}
\pmcreated{2013-03-22 17:09:21}
\pmmodified{2013-03-22 17:09:21}
\pmowner{mps}{409}
\pmmodifier{mps}{409}
\pmtitle{nonogram}
\pmrecord{8}{39465}
\pmprivacy{1}
\pmauthor{mps}{409}
\pmtype{Definition}
\pmcomment{trigger rebuild}
\pmclassification{msc}{68Q17}
\pmclassification{msc}{15A36}
\pmclassification{msc}{05B99}
\pmclassification{msc}{00A08}
\pmdefines{paint by number}
\pmdefines{griddler}
\pmdefines{another solution problem}
\pmdefines{ASP}

% this is the default PlanetMath preamble.  as your knowledge
% of TeX increases, you will probably want to edit this, but
% it should be fine as is for beginners.

% almost certainly you want these
\usepackage{amssymb}
\usepackage{amsmath}
\usepackage{amsfonts}

% used for TeXing text within eps files
%\usepackage{psfrag}
% need this for including graphics (\includegraphics)
\usepackage{graphicx}
% for neatly defining theorems and propositions
%\usepackage{amsthm}
% making logically defined graphics
%%%\usepackage{xypic}

% there are many more packages, add them here as you need them

% define commands here

\begin{document}
A \emph{nonogram}, also known as a \emph{paint by numbers puzzle} or a
\emph{griddler}, is a puzzle in which the solver must reconstruct a
picture drawn in a grid using clues which give the distribution of
pixels in each row and column.

The simplest nonograms use two colors, white (usually the background)
and black (usually the foreground).  Here is a sample black and white
nonogram:
\begin{figure}[hh]
\begin{center}
\includegraphics{initial.epsi}
\caption{A nonogram in its unsolved state.}
\end{center}
\end{figure}
The numbers to the left of a row indicate how many runs of black
squares there are in that row and how long each such run is.  The
numbers above a column serve the same purpose for that column.

For instance, the clue for the first row is $(1,1)$.  Since there are
six different ways of selecting two nonadjacent squares in a row of
length five, we cannot yet conclude anything from this clue.  The next
clue, $(5)$, indicates that the second row is completely filled in
with black squares.  Such clues are rare on real puzzles.  Slightly
more common are clues such as the fourth row clue $(3)$.  Although we
cannot yet know precisely where the run of three black squares is
positioned, by the pigeonhole principle the middle square of that row
must be black.  Finally, since two runs of black must be separated by
at least one square of another color, the clue $(2,2)$ tells us that
the fifth row has two black squares followed by a white square and
then two more black squares. 

Applying the information we have so far yields the following picture.
\begin{figure}[hh]
\begin{center}
\includegraphics{stage1.epsi}
\caption{The nonogram after some clues have been applied.}
\end{center}
\end{figure}
With some of the squares filled in, the remaining clues give us more
useful information.  For instance, the first vertical clue says that
the first column has exactly two black squares, and we have already
drawn two squares in that column, so the other three squares must all
be white.  By using the positions of the squares in the second and
fourth columns we can finish those columns as well.  By using only
vertical clues, we can finish every column but the last.
\begin{figure}[hh]
\begin{center}
\includegraphics{stage2.epsi}
\caption{The nonogram in a nearly complete state.}
\end{center}
\end{figure}

To complete the puzzle, we appeal to the fact that the first row must
have two black squares while the third row only has one.  In the fifth
row we therefore place a black square in the first position and a
white square in the third position.
\begin{figure}[hh]
\begin{center}
\includegraphics{stage3.epsi}
\caption{A solved nonogram.}
\end{center}
\end{figure}
We are done.

Given a picture, it is straightforward to generate the clues for that
picture: simply determine for each row and column the lengths of the
runs of the non-background colors.  However, it is not always
straightforward to recover the picture from the clues, and a designer
can exploit this fact to make puzzles of varying levels of difficulty.
A challenge facing the designer is the fact that not every puzzle has
a unique solution.  In particular, if we view solutions as rectangular
matrices with entries $0$ and $1$, where $0$ represents the background
color and $1$ the foreground color, then any 5-by-5 permutation matrix
is a solution to the following puzzle.
\begin{figure}[hh]
\begin{center}
\includegraphics{ambiguous.epsi}
\caption{An ambiguous nonogram.}
\end{center}
\end{figure}
Logic puzzles with multiple solutions can be interesting for
mathematical reasons, but they are unacceptable as puzzles.  The
problem of determining whether a puzzle has a unique solution is
sometimes called the \emph{Another Solution Problem}, or \emph{ASP},
and is known to be NP-complete for many puzzles, including nonograms.

% mention generalizations:
%   (1) more colors 
%   (2) more general graphs than grid graphs (arrangements of pseudolines?) 

Nonograms as we have described them can be generalized by allowing
more colors to be used in the picture.  The clues then must indicate
both which non-background colors appear in which order in a row or
column and how long each run is.  Another generalization is to use a
non-square grid.  For example, there are \emph{triddlers}, which are
nonograms on a grid tiled by equilateral triangles and which therefore
require clues for three directions.  One could also imagine
generalizations to higher dimensions, leading to building-block
sculpture puzzles and the like.

\begin{thebibliography}{99}
\bibitem{UeNa}
N.~Ueda and T.~Nagao. \emph{NP-completeness results for NONOGRAM via
parsimonious reductions.} Technical Report TR96-0008, Department of
Computer Science, Tokyo Institute of Technology, 1996.

\bibitem{Ya}
T.~Yato. \emph{Complexity and completeness of finding another solution
and its application to puzzles}.  Thesis in information science, the
University of Tokyo, 2003
\end{thebibliography}

\PMlinkescapeword{complete}
\PMlinkescapeword{distribution}
\PMlinkescapeword{generate}
\PMlinkescapeword{information}
\PMlinkescapeword{levels}
\PMlinkescapeword{logic}
\PMlinkescapeword{multiple}
\PMlinkescapeword{order}
\PMlinkescapeword{place}
\PMlinkescapeword{real}
\PMlinkescapeword{represents}
\PMlinkescapeword{separated}
\PMlinkescapeword{solution}
\PMlinkescapeword{solutions}
%%%%%
%%%%%
\end{document}
