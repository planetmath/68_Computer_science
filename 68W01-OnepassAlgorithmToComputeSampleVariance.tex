\documentclass[12pt]{article}
\usepackage{pmmeta}
\pmcanonicalname{OnepassAlgorithmToComputeSampleVariance}
\pmcreated{2013-03-22 16:45:19}
\pmmodified{2013-03-22 16:45:19}
\pmowner{stevecheng}{10074}
\pmmodifier{stevecheng}{10074}
\pmtitle{one-pass algorithm to compute sample variance}
\pmrecord{9}{38981}
\pmprivacy{1}
\pmauthor{stevecheng}{10074}
\pmtype{Algorithm}
\pmcomment{trigger rebuild}
\pmclassification{msc}{68W01}
\pmclassification{msc}{65-00}
\pmclassification{msc}{62-00}

% The standard font packages
\usepackage{amssymb}
\usepackage{amsmath}
\usepackage{amsfonts}

% For neatly defining theorems and definitions
%\usepackage{amsthm}

% Including EPS/PDF graphics (\includegraphics)
%\usepackage{graphicx}

% Making matrix-based graphics
%%%\usepackage{xypic}

% Enumeration lists with different styles
%\usepackage{enumerate}

% Set up the theorem environments
%\newtheorem{thm}{Theorem}
%\newtheorem*{thm*}{Theorem}

\providecommand{\defnterm}[1]{\emph{#1}}

% The standard number systems
\newcommand{\complex}{\mathbb{C}}
\newcommand{\real}{\mathbb{R}}
\newcommand{\rat}{\mathbb{Q}}
\newcommand{\nat}{\mathbb{N}}
\newcommand{\intset}{\mathbb{Z}}

% Absolute values and norms
% Normal, wide, and big versions of the delimeters
\providecommand{\abs}[1]{\lvert#1\rvert}
\providecommand{\absW}[1]{\left\lvert#1\right\rvert}
\providecommand{\absB}[1]{\Bigl\lvert#1\Bigr\rvert}
\providecommand{\norm}[1]{\lVert#1\rVert}
\providecommand{\normW}[1]{\left\lVert#1\right\rVert}
\providecommand{\normB}[1]{\Bigl\lVert#1\Bigr\rVert}

\newcommand{\xb}{\overline{x}}

\begin{document}
\PMlinkescapeword{derivation}
\PMlinkescapeword{terms}
\PMlinkescapeword{formula}

In many situations it is desirable to calculate,
in one iteration, the sample variance of many numbers, and without 
having to have every number available
in computer memory before beginning processing.

Let $x_1, x_2, \dotsc$ denote the data.  
The na\"ive formula for calculating the sample
variance in one pass,
\[
v = \frac{1}{n-1} \sum_{i=1}^n (x_i - \xb)^2
= \frac{\left( \sum_{i=1}^n x_i^2 \right) - n\xb^2}{n-1}
\,, \quad \xb = \frac{1}{n} \sum_{i=1}^n x_i\,,
\]
suffers from computational round-off error.
If the mean $\xb$ is large in absolute value,
and $\sum_{i=1}^n x_i^2$ is close to $n \xb^2$,
then the subtraction at the end will tend to lose
significant digits on the result.
Also, in rare cases, the sum of squares $\sum_{i=1}^n x_i^2$
can overflow on a computer.

A better alternative, though requiring more work per iteration,
 is to calculate the running sample mean
and variance instead,
and update these as each datum is processed.
Here we give the derivation of the one-pass algorithm --- which 
involves nothing more than simple algebraic manipulations.

Define the running arithmetic mean
and the sum of squared residuals:
\[
a_n = \frac{1}{n} \sum_{i=1}^n x_i\,,
\quad
s_n = \sum_{i=1}^n (x_i - a_n)^2\,.
\]
We want to express $a_{n+1}$ and $s_{n+1}$
in terms of the old values $a_n$ and $s_n$.

For convenience, let $\delta = x_{n+1} - a_n$
and $\gamma = a_{n+1} - a_n$.
Then we have
\[
a_{n+1} = \frac{n a_n + x_{n+1}}{n+1} = 
\frac{(n+1)a_n + x_{n+1} - a_n}{n+1}
= a_n + \frac{\delta}{n+1}\,.
\]

For the variance calculation,
we have
\begin{align*}
s_{n+1} &= \sum_{i=1}^n \bigl( (x_i - a_n) - \gamma \bigr)^2
+ (x_{n+1} - a_{n+1})^2
\\
&= \sum_{i=1}^n (x_i - a_n)^2 - 2\gamma \sum_{i=1}^n (x_i - a_n)
+ \sum_{i=1}^n \gamma^2 + (x_{n+1} - a_{n+1})^2 \\
&= s_n + 0 + n \gamma^2 + (x_{n+1} - a_{n+1})^2\,.
\end{align*}
Now observe:
\begin{align*}
\gamma = \frac{\delta}{n+1}\,,
\quad
x_{n+1} - a_{n+1} = \delta - \gamma = (n+1) \gamma - \gamma = n\gamma\,;
\end{align*}
hence we obtain:
\begin{align*}
s_{n+1} = s_n + n\gamma^2 + n^2 \gamma^2
= s_n + n(n+1) \gamma^2
&= s_n + n \gamma \delta  \\
&= 
s_n + 
( x_{n+1} - a_{n+1}) \delta\,.
\end{align*}

Note that the number to be added to $s_n$ is never negative,
so no cancellation error will occur from this procedure.
(However, there can still be computational round-off error if $s_{n+1} - s_n$
happens to be very small compared to $s_n$.)

\medskip

The recurrence relation for the 
sample covariance of two lists of numbers $x_1, x_2, \dotsc$
and $y_1, y_2, \dotsc$
can be derived similarly.
If $a_n$ and $b_n$ denote the arithmetic means
of first $n$ numbers of each of the two lists respectively,
then the sum of adjusted products
\[
c_n = \sum_{i=1}^n (x_i - a_n)(y_i - b_n)
\]
can be incrementally updated by
\[
c_{n+1} = c_n + (y_{n+1} - b_{n+1}) \, (x_{n+1} - a_n)
= c_n + (x_{n+1} - a_{n+1}) \, (y_{n+1} - b_n)\,.
\]

\begin{thebibliography}{3}
\bibitem{Welford}
B. P. Welford. ``Note on a Method for Calculating Corrected Sums of Squares and Products''. {\it Technometrics}, Vol. 4, No. 3 (Aug., 1962), p. 419-420.

\bibitem{Wiki}
``\PMlinkexternal{Algorithms for calculating variance}{http://en.wikipedia.org/wiki/Algorithms\_for\_calculating\_variance}''. {\it Wikipedia, The Free Encyclopedia}.  Accessed 25 February 2007.
\end{thebibliography}

%%%%%
%%%%%
\end{document}
