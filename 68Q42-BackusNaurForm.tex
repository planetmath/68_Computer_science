\documentclass[12pt]{article}
\usepackage{pmmeta}
\pmcanonicalname{BackusNaurForm}
\pmcreated{2013-03-22 17:37:00}
\pmmodified{2013-03-22 17:37:00}
\pmowner{CWoo}{3771}
\pmmodifier{CWoo}{3771}
\pmtitle{Backus-Naur form}
\pmrecord{6}{40034}
\pmprivacy{1}
\pmauthor{CWoo}{3771}
\pmtype{Definition}
\pmcomment{trigger rebuild}
\pmclassification{msc}{68Q42}
\pmclassification{msc}{68Q45}
\pmsynonym{BNF}{BackusNaurForm}
\pmsynonym{Backus normal form}{BackusNaurForm}

\usepackage{amssymb,amscd}
\usepackage{amsmath}
\usepackage{amsfonts}
\usepackage{mathrsfs}

% used for TeXing text within eps files
%\usepackage{psfrag}
% need this for including graphics (\includegraphics)
%\usepackage{graphicx}
% for neatly defining theorems and propositions
\usepackage{amsthm}
% making logically defined graphics
%%\usepackage{xypic}
\usepackage{pst-plot}
\usepackage{psfrag}

% define commands here
\newtheorem{prop}{Proposition}
\newtheorem{thm}{Theorem}
\newtheorem{ex}{Example}
\newcommand{\real}{\mathbb{R}}
\newcommand{\pdiff}[2]{\frac{\partial #1}{\partial #2}}
\newcommand{\mpdiff}[3]{\frac{\partial^#1 #2}{\partial #3^#1}}
\begin{document}
\PMlinkescapeword{expression}

The \emph{Backus-Naur form} (or \emph{BNF} as it is commonly denoted) is a convenient notation used to represent context-free grammars in an intuitive and more compact manner.  In a Backus-Naur form, there are only four symbols that have special meaning: 
$$\verb.<.\qquad \verb.>.\qquad \verb.::=.\qquad \verb.|.$$ 
Given a context-free grammar $(\Sigma,N,P,S)$, a non-terminal (a symbol in the alphabet $N$) is always enclosed in \verb.<. and \verb.>. (e.g. \verb.<expression>.).
A terminal (a symbol in the alphabet $\Sigma$) is often represented as itself, though in the context of computer languages a terminal symbol is often enclosed in single quotes. A production $(\textit{non-terminal}\to \textit{symbols})$ in $P$ is then represented as

\begin{center}
\verb.<.\textit{non-terminal}\verb.> ::= .\textit{symbols}
\end{center}

The symbol \verb.|. is used in BNF to combine multiple productions in $P$ into
one rule. For instance, if $P := \left\{S\to A, S\to B\right\}$, then $P$ in
BNF is

\begin{center}
\verb.<.\textit{S}\verb.> ::= .\textit{A} \verb.|. \textit{B}
\end{center}

\textbf{Examples}.
\begin{itemize}
\item
Let $\Sigma=\lbrace a,b,c\rbrace$, $N=\lbrace S,T,U\rbrace$ be the terminal and non-terminal alphabets of a formal grammar, and $$P=\lbrace S\to aSb, S\to TU, S\to c, T\to cUc, T\to ac, U\to bT, U\to cb\rbrace$$ is the set of productions.  Then $(\Sigma,N,P,S)$ is a context-free grammar.  The BNF for $P$ is 
\begin{eqnarray*}
\verb.<.\textit{S}\verb.>.&\verb.:=.& \textit{a}\verb.<.\textit{S}\verb.>.\textit{b}\verb. | <.\textit{T}\verb.><.\textit{U}\verb.> | .\textit{c}\\
\verb.<.\textit{T}\verb.>.&\verb.:=.& \textit{c}\verb.<.\textit{U}\verb.>.\textit{c} \verb. | .\textit{ac}\\
\verb.<.\textit{U}\verb.>.&\verb.:=.& \textit{b}\verb.<.\textit{T}\verb.> | .\textit{cb}
\end{eqnarray*}
\item
For another example, let us transform the context-free grammar specified in the \PMlinkname{parent entry}{ContextFreeLanguage} to BNF.
For readability, we will call $S$ \textit{expression}, $A$ \textit{term},
$B$ \textit{factor}, $C$ \textit{number}, and $D$ \textit{digit}. The
BNF for $P$ is then

\begin{eqnarray}
\verb.<.expression\verb.>. & \verb.::=. & \verb.<.term\verb.>.\,
\verb.|.\,\verb.<.expression\verb.>.\,\verb.+.\,\verb.<.term\verb.>.\,
\verb.|.\,\verb.<.expression\verb.>.\,\verb.-.\,\verb.<.term\verb.>. \nonumber \\
\verb.<.term\verb.>. & \verb.::=. & \verb.<.factor\verb.>.\,
\verb.|.\,\verb.<.term\verb.>.\,\verb.*.\,\verb.<.factor\verb.>.\,
\verb.|.\,\verb.<.term\verb.>.\,\verb./.\,\verb.<.factor\verb.>. \nonumber \\
\verb.<.factor\verb.>. & \verb.::=. & \verb.<.number\verb.>.\,\verb.|.\, \verb.(<.expression\verb.>). \nonumber \\
\verb.<.number\verb.>. & \verb.::=. & \verb.<.digit\verb.>.\,\verb.|.\, \verb.<.number\verb.>.\,\verb.<.digit\verb.>. \nonumber \\
\verb.<.digit\verb.>. & \verb.::=. & \verb.0 | 1 | 2 | 3 | 4 | 5 | 6 | 7 | 8 | 9.\label{b}
\end{eqnarray}
\end{itemize}

\textbf{Remark}.  As the syntaxes of most programming languages are context-free grammars (or very close to it), the Backus-Naur form can be used to specify these syntaxes.  In fact, BNF was invented to specify the syntax of ALGOL 60.
%%%%%
%%%%%
\end{document}
