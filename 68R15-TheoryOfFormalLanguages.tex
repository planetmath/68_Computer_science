\documentclass[12pt]{article}
\usepackage{pmmeta}
\pmcanonicalname{TheoryOfFormalLanguages}
\pmcreated{2013-03-22 15:06:37}
\pmmodified{2013-03-22 15:06:37}
\pmowner{rspuzio}{6075}
\pmmodifier{rspuzio}{6075}
\pmtitle{theory of formal languages}
\pmrecord{16}{36843}
\pmprivacy{1}
\pmauthor{rspuzio}{6075}
\pmtype{Topic}
\pmcomment{trigger rebuild}
\pmclassification{msc}{68R15}
\pmclassification{msc}{68Q70}
\pmclassification{msc}{68Q45}
\pmclassification{msc}{68Q42}
\pmclassification{msc}{20M05}
\pmclassification{msc}{20F10}
\pmclassification{msc}{08A50}
\pmclassification{msc}{03D40}
\pmclassification{msc}{03D05}
\pmclassification{msc}{03C07}

% this is the default PlanetMath preamble.  as your knowledge
% of TeX increases, you will probably want to edit this, but
% it should be fine as is for beginners.

% almost certainly you want these
\usepackage{amssymb}
\usepackage{amsmath}
\usepackage{amsfonts}

% used for TeXing text within eps files
%\usepackage{psfrag}
% need this for including graphics (\includegraphics)
%\usepackage{graphicx}
% for neatly defining theorems and propositions
%\usepackage{amsthm}
% making logically defined graphics
%%%\usepackage{xypic}

% there are many more packages, add them here as you need them

% define commands here
\begin{document}
Note: This entry is very rough at the moment, and requires work. I mainly wrote it to help motivate other entries and to organize entries on this topic and point out holes in our coverage.  Right now, it is mainly a list of entries, many of which have not been written yet.  Under the first heading, there is short paragraph.  Eventually, there should be such a paragraph under each entry and a bibliography at the end.  However, this is a lot of work for one person, so this entry is world editable in the hope that others who are knowledgable in this topic will contribute their expertise.

\section{Basic concepts and terminology}

Loosely speaking, a formal language is a language whose structrure can be specified with mathematical precision.  The study of formal languages is not only interesting as a mathematical discipline in its own right, but also because of its relevance to the foundations of mathematics, its applications, and surprising connections with other branches of mathematics.

\begin{itemize}
\item alphabet
\item automaton
\item concatenation
\item derivation
\item grammar
\item homomorphism of languages
\item isomorphism of languages
\item language
\item abstract family of languages (AFL)
\item reversal or mirror image
\item initial symbol
\item production
\item rewriting rule
\item semantics
\item syntax
\item terminal symbol
\item non-terminal symbol
\item word
\end{itemize}

\section{Classification of languages}
\begin{itemize}
\item Chomsky Hierarchy
\item regular language
\item context-free language
\item context-sensitive language
\item phrase-structure language
\item type-3 language
\item type-2 language
\item type-1 language
\item type-0 language
\end{itemize}

\section{Regular (type 3) languages}
\begin{itemize}
\item regular expression
\item Kleene star
\item Kleene algebra
\item left-linear grammar
\item right-linear grammar
\item finite automaton
\end{itemize}

\section{Context-free (type 2) languages}
\begin{itemize}
\item Chomsky normal form
\item derivation tree
\item intersection of context-free and regular languages is context-free
\item leftmost derivation
\item rightmost derivation
\item Greibach normal form
\item pumping lemma
\item pushdown automaton
\item deterministic pushdown automaton
\item a language is context-free iff it can be recognized by a pushdown automaton
\item Earley's algorithm
\item ambiguous grammar
\item \PMlinkname{$LL(k)$}{LLk} grammar
\item left-factored grammar
\item \PMlinkname{$LR(k)$}{LRk} grammar
\item every $LR(k)$ grammar can be recognized by a deterministic pushdown 
automaton 
\item every language which can be recognized by a deterministic pushdown 
automaton can be described by an $LR(1)$ grammar
\end{itemize}

\section{Context-sensitive (type 1) languages}
\begin{itemize}
\item Kuroda normal form
\item length-increasing grammar
\item a language is context-sensitive iff it can be generated by a length-increasing grammar
\item linear bounded automaton
\item a language is context-sensitive iff it can be recognized by a linear bounded automaton
\end{itemize}

\section{Phrase-structure (type 0) languages}
\begin{itemize}
\item recursive language
\item recursively enumerable language, co-recursively enumerable language
\item language that is neither recursively enumerable, nor co-recursively enumerable
\item every phrase-structure language is recursively enumerable
\end{itemize}

\section{Other types of languages and automata that describe them}
\begin{itemize}
\item star-free language versus aperiodic finite automaton
\item a star-free language is regular, but not conversely
\item mildly context-sensitive language versus embedded pushdown automaton
\item tree-adjoining grammar
\item languages generated by tree-adjoining grammars are exactly the mildly context-sensitive languages
\item a context-free language is mildly context-sensitive, but not conversely
\item indexed language versus nested stack automaton
\item a mildly context-sensitive language is indexed, but not conversely
\item an indexed language is context-sensitive, but not conversely
\end{itemize}

\section{Connection to group and semigroup theory}
\begin{itemize}
\item finitely presented group
\item automatic group
\item \PMlinkname{semi-Thue system}{SemiThueSystem}
\item Post system
\item word problem
\item Post correspondence problem
\item conjugacy problem
\end{itemize}

\section{Decidability}
\begin{itemize}
\item membership problem
\item emptiness problem
\item recursively enumerable language
\item recursive language
\end{itemize}

\section{Special languages}
\begin{itemize}
\item Dyck language
\item derivation language
\end{itemize}
%%%%%
%%%%%
\end{document}
